\documentclass[12pt]{book}
\usepackage{color}
\usepackage{fancyvrb}
\newcommand{\VerbBar}{|}
\newcommand{\VERB}{\Verb[commandchars=\\\{\}]}
\DefineVerbatimEnvironment{Highlighting}{Verbatim}{commandchars=\\\{\}}
% Add ',fontsize=\small' for more characters per line
\usepackage{framed}
\definecolor{shadecolor}{RGB}{248,248,248}
\newenvironment{Shaded}{\begin{snugshade}}{\end{snugshade}}
\newcommand{\AlertTok}[1]{\textcolor[rgb]{0.94,0.16,0.16}{#1}}
\newcommand{\AnnotationTok}[1]{\textcolor[rgb]{0.56,0.35,0.01}{\textbf{\textit{#1}}}}
\newcommand{\AttributeTok}[1]{\textcolor[rgb]{0.77,0.63,0.00}{#1}}
\newcommand{\BaseNTok}[1]{\textcolor[rgb]{0.00,0.00,0.81}{#1}}
\newcommand{\BuiltInTok}[1]{#1}
\newcommand{\CharTok}[1]{\textcolor[rgb]{0.31,0.60,0.02}{#1}}
\newcommand{\CommentTok}[1]{\textcolor[rgb]{0.56,0.35,0.01}{\textit{#1}}}
\newcommand{\CommentVarTok}[1]{\textcolor[rgb]{0.56,0.35,0.01}{\textbf{\textit{#1}}}}
\newcommand{\ConstantTok}[1]{\textcolor[rgb]{0.00,0.00,0.00}{#1}}
\newcommand{\ControlFlowTok}[1]{\textcolor[rgb]{0.13,0.29,0.53}{\textbf{#1}}}
\newcommand{\DataTypeTok}[1]{\textcolor[rgb]{0.13,0.29,0.53}{#1}}
\newcommand{\DecValTok}[1]{\textcolor[rgb]{0.00,0.00,0.81}{#1}}
\newcommand{\DocumentationTok}[1]{\textcolor[rgb]{0.56,0.35,0.01}{\textbf{\textit{#1}}}}
\newcommand{\ErrorTok}[1]{\textcolor[rgb]{0.64,0.00,0.00}{\textbf{#1}}}
\newcommand{\ExtensionTok}[1]{#1}
\newcommand{\FloatTok}[1]{\textcolor[rgb]{0.00,0.00,0.81}{#1}}
\newcommand{\FunctionTok}[1]{\textcolor[rgb]{0.00,0.00,0.00}{#1}}
\newcommand{\ImportTok}[1]{#1}
\newcommand{\InformationTok}[1]{\textcolor[rgb]{0.56,0.35,0.01}{\textbf{\textit{#1}}}}
\newcommand{\KeywordTok}[1]{\textcolor[rgb]{0.13,0.29,0.53}{\textbf{#1}}}
\newcommand{\NormalTok}[1]{#1}
\newcommand{\OperatorTok}[1]{\textcolor[rgb]{0.81,0.36,0.00}{\textbf{#1}}}
\newcommand{\OtherTok}[1]{\textcolor[rgb]{0.56,0.35,0.01}{#1}}
\newcommand{\PreprocessorTok}[1]{\textcolor[rgb]{0.56,0.35,0.01}{\textit{#1}}}
\newcommand{\RegionMarkerTok}[1]{#1}
\newcommand{\SpecialCharTok}[1]{\textcolor[rgb]{0.00,0.00,0.00}{#1}}
\newcommand{\SpecialStringTok}[1]{\textcolor[rgb]{0.31,0.60,0.02}{#1}}
\newcommand{\StringTok}[1]{\textcolor[rgb]{0.31,0.60,0.02}{#1}}
\newcommand{\VariableTok}[1]{\textcolor[rgb]{0.00,0.00,0.00}{#1}}
\newcommand{\VerbatimStringTok}[1]{\textcolor[rgb]{0.31,0.60,0.02}{#1}}
\newcommand{\WarningTok}[1]{\textcolor[rgb]{0.56,0.35,0.01}{\textbf{\textit{#1}}}}
\usepackage[nottoc]{tocbibind}
\usepackage{xspace}
\usepackage[utf8]{inputenc}
\usepackage{titling}
\usepackage{sectsty}
\usepackage[T1]{fontenc}
%\usepackage[ngerman]{babel}     %deutsche Silbentrennung
\usepackage[english]{babel}     %deutsche Silbentrennung
%\usepackage[automark]{scrpage2}
\usepackage{textcomp}
\usepackage{verbatim}
\PassOptionsToPackage{hyphens}{url}\usepackage[colorlinks=true, linkcolor=blue, citecolor=red, urlcolor=blue]{hyperref}
\makeatletter
\g@addto@macro{\UrlBreaks}{\UrlOrds}
\makeatother
\usepackage{fancyhdr}
\usepackage{soul}
\usepackage{makeidx}

\usepackage{amsmath}
\usepackage{amssymb}
%\usepackage{url}
%\usepackage{cite}
\usepackage[left=2.5cm, right=2.5cm, top=2.5cm, bottom=2.5cm]{geometry}
\usepackage{graphicx}
\usepackage{url}
\usepackage{float}
\usepackage{setspace}
\usepackage{subfigure}
\usepackage{titleref}
\usepackage[sort&compress]{natbib}
%\usepackage[round,authoryear]{natbib}
\numberwithin{equation}{chapter}
\usepackage[bf]{caption}
\usepackage{multirow}
\usepackage{longtable}
\usepackage{booktabs} % 三线表
\usepackage[figuresright]{rotating}
\usepackage{multicol} %用于实现在同一页中实现不同的分栏
\usepackage{array} % 表格居中并且能够设置宽度
% 压缩pdf体积
%\pdfcompresslevel=9
%\pdfminorversion=5
%\pdfobjcompresslevel=3


\newcommand{\tabincell}[2]{\begin{tabular}{@{}#1@{}}#2\end{tabular}}
\renewcommand{\captionfont}{\small\slshape}
\renewcommand{\contentsname}{Contents}
\newenvironment{bottompar}{\par\vspace*{\fill}}{\clearpage}% move to the end of the page
%\renewcommand{\contentsname}{TABLE OF CONTENTS}
\addto\captionsenglish{% Replace "english" with the language you use
  \renewcommand{\contentsname}%
    {TABLE OF CONTENTS}%
}
\providecommand{\tightlist}{%
  \setlength{\itemsep}{0pt}\setlength{\parskip}{0pt}}


\renewcommand{\headrulewidth}{0pt}
\pagestyle{fancy}
\setlength\bibindent{4em}

\begin{document}

%% -----------------------------------------------------------------------------
%% Title page
%% -----------------------------------------------------------------------------
\begin{titlepage}
\vspace{1cm}
\begin{center}
\linespread{2}\normalsize \bfseries \MakeUppercase{Predicting Truck Drivers' Critical Events: Efficient Bayesian Hierarchical Models}
\end{center}

\vspace{7cm}

\begin{center}
{Miao Cai,  M.S.}
\vspace{0.3cm}

Draft on \today

\vspace{9cm}

Dissertation Presented to the Graduate Faculty of \\
Saint Louis University in Partial Fulfillment \\
of the Requirements for the Degree of\\
Public Health Studies, Ph.D.\\
\vspace{.5cm}
\the\year
\end{center}

\end{titlepage}
\clearpage

\pagenumbering{roman}

\begin{bottompar}
\begin{center}
\textcopyright \xspace Copyright by\\
Miao Cai \\
ALL RIGHTS RESERVED\\
\vspace{.5cm}
\the\year
\end{center}
\end{bottompar}

\topskip0pt
\vspace*{\fill}
\underline{COMMITTEE IN CHARGE OF CANDIDACY:}

\vspace{.3cm}
Professor Steven E. Rigdon, Ph.D.\\
{\setlength{\parindent}{20ex}  \indent \underline{Chairperson and Advisor}}

\vspace{.3cm}
Professor Hong Xian, Ph.D.

\vspace{.3cm}
Assistant Professor Fadel Megahed, Ph.D.
\vspace*{\fill}








\hypertarget{dedication}{%
\chapter*{Dedication}\label{dedication}}
\addcontentsline{toc}{chapter}{Dedication}

I dedicate this dissertation to my parents, Zhimin Cai and Guizhen Xu, who believe in the power of higher education, hard work, and always support me.

\hypertarget{acknowledgement}{%
\chapter*{Acknowledgement}\label{acknowledgement}}
\addcontentsline{toc}{chapter}{Acknowledgement}

I want to thank my PhD mentor and committee chair Dr.~Steven E. Rigdon, committee members Dr.~Hong Xian and Dr.~Fadel Megahed.

\cleardoublepage
\tableofcontents

\listoffigures
\listoftables

\mainmatter

\doublespacing

\hypertarget{introduction}{%
\chapter{INTRODUCTION}\label{introduction}}

\hypertarget{transportation-safety}{%
\section{Transportation safety}\label{transportation-safety}}

Traffic safety \index{traffic safety} is a pressing public health issue that involves huge lives losses and financial burden across the world and in the United States (US).
As reported by the World Health Organization (WHO \protect\hyperlink{ref-who2018}{2018}\protect\hyperlink{ref-who2018}{b}), road injury was the eighth cause of death globally in 2016, killing approximately 1.4 million people, which consisted of about 2.5\% of all deaths in the world.
If no sustained action is taken, road injuries are predicted to be the seventh leading cause of death across the world by 2030 (WHO \protect\hyperlink{ref-who2018b}{2018}\protect\hyperlink{ref-who2018b}{a}).
Compared to the victims who were claimed lives by diseases, people killed in traffic are mostly early- or middle-aged, particularly those aged 4 to 44 years old (Litman \protect\hyperlink{ref-litman2013transportation}{2013}; Evans \protect\hyperlink{ref-evans2014traffic}{2014}).
Without traffic accidents, these victims could have much longer lives with normal health.

Apart from fatal deaths, road traffic injuries were also reported to be the cause of 50 million non-fatal life injuries and approximately 75.5 million disability-adjusted life years globally (Staton et al. \protect\hyperlink{ref-staton2016road}{2016}).
In high-income countries, the majority of non-death costs were attributable to non-fatal crashes, with 2\% of non-fatal events leading to over 40\% of life-time medical costs (Ameratunga, Hijar, and Norton \protect\hyperlink{ref-ameratunga2006road}{2006}).
Besides non-fatal injuries, traffic safety is a major economic burden.
The global economic losses attributable to transportation safety were estimated to be 518 billion the United States Dollars (USD), which accounted for 1\% the gross domestic product (GDP) in low-income countries, 1.5\% in middle-income countries, and 2\% in high-income countries (Peden et al. \protect\hyperlink{ref-peden2004world}{2004}; Dalal et al. \protect\hyperlink{ref-dalal2013economics}{2013}).

Specifically in the United States, transportation contributed to the highest number of fatal occupational injuries, leading to 2,077 deaths and accounting for over 40\% of all fatal occupational injuries in 2017 (The United States, Bureau of Labor Statistics \protect\hyperlink{ref-bols}{2017}).
The National Safety Council reported that the number of deaths attributable to car crashes will be at least 40,000 in 2018, which is the third straight year that this number is over 40,000 (The National Safety Council \protect\hyperlink{ref-nsc2018}{2018}).
A comparison study of 26 developed countries revealed that 20 to 60 traffic deaths per billion kilometers were reduced from 2011 (Evans \protect\hyperlink{ref-evans2014traffic}{2014}).
Even though fatality rates attributable to road traffic in the US were reduced by 40\% during that period, the rates declined more rapidly in all other 25 countries.
Given the large amounts of investments in roads, improved vehicle protection and traffic policy implementation, and advanced emergency and trauma care, the reduction in traffic associated fatality rates is nominal (Litman \protect\hyperlink{ref-litman2013transportation}{2013}).
If the change of traffic fatality rates match those in other unremarkable countries, 20,000 traffic deaths could have been prevented each year (Evans \protect\hyperlink{ref-evans2014traffic}{2014}).

The impact of road injuries is even more impactful in developing countries than in developed countries (Goonewardene et al. \protect\hyperlink{ref-goonewardene2010road}{2010}).
Although low- and middle-income countries own only 48\% of the registered vehicles in the world, 90\% of road traffic fatalities and injuries were estimated to occur in these countries, which continue to be escalating due to rapid urbanization and motorization (Dalal et al. \protect\hyperlink{ref-dalal2013economics}{2013}; Staton et al. \protect\hyperlink{ref-staton2016road}{2016}).
Ten developing countries, including Brazil, Cambodia, China, Egypt, India, Kenya, Mexico, Russia, Turkey, and Vietnam, account for almost half of all the road traffic in the world (Hyder et al. \protect\hyperlink{ref-hyder2012addressing}{2012}).
For example, China has around 100,000 traffic-related fatalities each year (Zhang et al. \protect\hyperlink{ref-zhang2010road}{2010}), accounting for around 80\% of all accidental deaths, with 87\% of them were caused by motor vehicles in 2015 (Jiang et al. \protect\hyperlink{ref-jiang2017transport}{2017}).
In comparison, 148,707 lives were claimed by road collisions in India in 2015, with the road fatality rate similar to the global average level of 17.4 deaths per 100,000 people (National Crime Records Bureau, Government of India \protect\hyperlink{ref-india2015}{2015}).

\hypertarget{trucking-safety}{%
\section{Trucking safety}\label{trucking-safety}}

In the US, the large commercial truck industry is the backbone of the economy. Approximately 70\% of freight is delivered via a truck at some point of their transportation, which account for 73.1\% of value and 71.3\% of volume of the domestic goods (Olson et al. \protect\hyperlink{ref-olson2016weight}{2016}; Anderson et al. \protect\hyperlink{ref-anderson2017exploratory}{2017}).
However, among all vehicles, large trucks are associated with more catastrophic accidents and therefore are the primary concern of traffic safety. In 2016, the Federal Motor Carrier Safety Administration (FMCSA) reported that 27\% fatal crashes in work zones involved large trucks (FMCSA \protect\hyperlink{ref-fmcsareport2016}{2018}). Among all 4,079 crashes involving large trucks or buses in 2016, 4,564 people (1.12 people per crash) were killed in the accidents (FMCSA \protect\hyperlink{ref-fmcsafacts2016}{2016}). Large truck crashes approximately claim 5,000 lives and cause 120,000 injuries each year, but only 15\% of these fatalities occur in the trucks, with a predominate 78\% occurred in the other vehicles (Neeley and Richardson Jr \protect\hyperlink{ref-neeley2009effect}{2009}). Besides, the economic losses associated with large truck crashes are also higher than those with passenger vehicles, with an estimated average cost of 91,000 US dollars per crash (Zaloshnja, Miller, and others \protect\hyperlink{ref-zaloshnja2008unit}{2008}).

The high risk of large trucks is attributed to two aspects of reasons (Huang et al. \protect\hyperlink{ref-huang2013development}{2013}). First, large truck drivers generally need to drive alone for long routes, under on-time demands, challenging weather and traffic conditions. Professional truck drivers usually need to work in shifts, and sometimes unavoidable late-night or early-morning shifts (Pylkkönen et al. \protect\hyperlink{ref-pylkkonen2015sleepiness}{2015}). These late-night or early-morning working shifts have been reported to be associated with sleep deprivation and disorders (Åkerstedt \protect\hyperlink{ref-aakerstedt1988sleepiness}{1988}; Mitler et al. \protect\hyperlink{ref-mitler1997sleep}{1997}; Solomon et al. \protect\hyperlink{ref-solomon2004healthcare}{2004}; Sallinen et al. \protect\hyperlink{ref-sallinen2005sleepiness}{2005}). Besides, commercial truck drivers are exposed to long route, constant concentration, and overtime work, which intertwines with sleep deprivation and disorder, and induce the fatigue symptoms among truck drivers. It is estimated that fatigue among long distance truck drivers caused up to 31\% of single vehicle fatal truck crashes (National Transportation Safety Board \protect\hyperlink{ref-ntsb1990}{1990}; Mitler et al. \protect\hyperlink{ref-mitler1997sleep}{1997}).

On the other hand, trucks have huge weights, large physical dimensions, and potentially carry hazardous cargoes. Although these huge-size trucks boost the transportation efficiency by increasing cargo capacity and reducing fuel costs per trip, they also raise public safety concerns (Lemp, Kockelman, and Unnikrishnan \protect\hyperlink{ref-lemp2011analysis}{2011}). Large trucks can weight up to 80,000 pounds by federal law, which are twenty times as much as a passenger vehicle (Department of Transportation, Utah \protect\hyperlink{ref-utah2019}{2019}). If these trucks travel at the speed of 65 miles per hour on the highway, it will take them 525 feet to stop, which is about two times the length of a football field (Department of Transportation, Utah \protect\hyperlink{ref-utah2019}{2019}). The large physical size also creates large blind spots on both sides of the truck, which poses more threat on smaller-sized vehicles. When a crash occurs between a large truck and a smaller vehicle, the sheer size and weight of the truck result in the tragedy that the victims are from the smaller vehicle instead of the trucks in around 80\% of the cases (Neeley and Richardson Jr \protect\hyperlink{ref-neeley2009effect}{2009}). In even worse case, commercial trucks crashes can cause massive casualties and regional public health emergency when the carried hazardous materials (such as gasoline and sulfuric acid) are leaked.

The importance of truck industry and the potential catastrophic consequences of truck crashes underscore the need to reduce crash risk and improve the safety of truck transportation.

\hypertarget{crashes-and-critical-events}{%
\section{Crashes and critical events}\label{crashes-and-critical-events}}

To reduce the lives and economic losses associated with trucks, numerous studies attempted to screen the risk factors for truck-related traffic crashes and make accurate prediction. Data used in these studies are characterized by rare events (dozens to thousands of times fewer crashes than non-crashes) (Theofilatos et al. \protect\hyperlink{ref-theofilatos2016predicting}{2016}, \protect\hyperlink{ref-theofilatos2018impact}{2018}). Therefore, the most common study design is a case-control study, matching a crash with one to up to ten non-crashes, and use statistical models such as logistic regressions to explain the causes or predict the crashes (Braver et al. \protect\hyperlink{ref-braver1997tractor}{1997}; C. Chen and Xie \protect\hyperlink{ref-chen2014modeling}{2014}; Meuleners et al. \protect\hyperlink{ref-meuleners2015obstructive}{2015}; Née et al. \protect\hyperlink{ref-nee2019road}{2019}). However, a case-control study is limited in estimating the incidence data. It may be contentious in selecting the ratio of controls to cases and how to select these controls (Grimes and Schulz \protect\hyperlink{ref-grimes2005compared}{2005}; Sedgwick \protect\hyperlink{ref-sedgwick2014case}{2014}). Due to the retrospective nature of crash data, it is also unrealistic to trace back to the real-time traffic, weather, and other environmental factors that were associated with the crashes.

Past truck safety literature almost exclusively focused on crashes, while ignoring the precursors to crashes. A precursor to crashes, also known as a critical events, adverse events, or near-miss crashes, is a pattern or signature associated with an increasing chance of truck crash (Saleh et al. \protect\hyperlink{ref-saleh2013accident}{2013}; Janakiraman, Matthews, and Oza \protect\hyperlink{ref-janakiraman2016discovery}{2016}). Truck critical events deserve more attention since they occur more frequently than crashes, suggest fatigue and a lapse in performance, and they can lead to giant crashes (Dingus et al. \protect\hyperlink{ref-dingus2006development}{2006}).

With the rapid development of modern technology, real-time driving and critical events data are recorded more and more by truck companies (Janakiraman, Matthews, and Oza \protect\hyperlink{ref-janakiraman2016discovery}{2016}). Although critical events do not always result in an accident, they could potentially be used as an early warning system to mitigate or prevent truck crashes (Kusano and Gabler \protect\hyperlink{ref-kusano2012safety}{2012}; Rome et al. \protect\hyperlink{ref-de2018near}{2018}).

\hypertarget{proposal}{%
\section{Proposal}\label{proposal}}

I propose using real-time truck ping data provided by the J.B. Hunt, real-time weather and traffic data to

\begin{enumerate}
\def\labelenumi{\arabic{enumi})}
\tightlist
\item
  quantify the association between truck crashes and critical events;
\item
  construct predictive models for truck critical events;
\item
  establish scalable Bayesian hierarchical models for large truck data.
\end{enumerate}

I believe that this work will contribute to statistical theories in constructing scalable Bayesian hierarchical statistical and reliability models using modern Markov chain Monte Carlo simulations. Realistically, these predictive models will inform policy-makers the functional relationship between driver characteristics, traffic, weather, and other real-time driving environment. These predictive models can be further used to provide data-driven justification to optimize trucking routes and minimize unsafe driving behaviors.

\hypertarget{literature-review}{%
\chapter{LITERATURE REVIEW}\label{literature-review}}

\hypertarget{precursors-to-crashes}{%
\section{Precursors to crashes}\label{precursors-to-crashes}}

\hypertarget{risk-factors}{%
\section{Risk factors}\label{risk-factors}}

\hypertarget{fatigue}{%
\subsection{Fatigue}\label{fatigue}}

Among all driver-related safety critical events, fatigue has become the most pressing problem of traffic accidents. It is estimated by National Sleep Foundation that approximately 32\% of drivers in U.S drive with fatigue over twice a month (National Sleep Foundation \protect\hyperlink{ref-nsleepf}{2008}). The American Automobile Association Foundation for Traffic Safety claimed that 16.5\% of fatal traffic accidents and 12.5\% of collisions related to injuries in the US in 2010 were associated with driving with fatigue (American Automobile Association Foundation for Traffic Safety \protect\hyperlink{ref-aaafoundation}{2010}).

(Anderson et al. \protect\hyperlink{ref-anderson2017exploratory}{2017})

Federal Motor Carrier Safety Administration (\protect\hyperlink{ref-hos2017}{2017}) demands that property-carrying drivers should not drive more than 14 consecutive hours after coming on duty after 10 hours of rest. This 14-hour restriction cannot be extended by off-duty time.

Drowsy driving is an especially common practice in less-developed countries because of cost control and tight schedule. Surveys of commercial and public road transportation companies in less-developed countries showed that employers were frequently forcing their employees to drive for longer hours and keep working even when they were exhausted (Zhang et al. \protect\hyperlink{ref-zhang2016traffic}{2016}; Odero, Khayesi, and Heda \protect\hyperlink{ref-odero2003road}{2003}; Nantulya and Reich \protect\hyperlink{ref-nantulya2002neglected}{2002}). High proportions of drowsy driving have been found among Brazilian (22\%) (Canani et al. \protect\hyperlink{ref-canani2005prevalence}{2005}), Argentinean (44\%) (Pérez-Chada et al. \protect\hyperlink{ref-perez2005sleep}{2005}), Pakistani (54\%) (Azam et al. \protect\hyperlink{ref-azam2014comparison}{2014}), and Thai (75\%) (Leechawengwongs et al. \protect\hyperlink{ref-leechawengwongs2006role}{2006}) truck or bus drivers. The mechanism of fatigue leading to safety critical events is that a driver's capability to stay alert to ambient traffic and pedestrians will be largely impaired. The reaction time is subsequently prolonged in that situation (Zhang et al. \protect\hyperlink{ref-zhang2014study}{2014}). It is estimated that 17 hours of continuous working lead to a deterioration of driving performance equivalent to a blood alcohol level of 0.05\% (MacLean, Davies, and Thiele \protect\hyperlink{ref-maclean2003hazards}{2003}). What makes the outcomes worse is that fatigue driving is more likely to happen on expressways and major highways where the speed limit is over 55 miles per hour (Knipling and Wang \protect\hyperlink{ref-knipling1994crashes}{1994}). This is especially concerning because fatigue driving safety critical events are more likely to result in serious injuries and fatalities, compared with non-fatigue driving safety critical events.

Stern et al. (\protect\hyperlink{ref-stern2018data}{2018}) reviewed the research related to fatigue of commercial motor vehicle drivers. Because of the difficulty of running a controlled experiment by imposing treatments, most research designs are observational studies, that is, they compare the effects of variables that are observed, not imposed. One exception to this is a \emph{randomized encouragement design} where drivers are randomized to receive some sort of incentive to apply some treatment, but are not forced to do so. If an effect is observed, we would conclude that it is due to the incentive, not necessarily to the actual treatment. Many studies use a cohort design or a case-control study. In a cohort design, a number of drivers is identified and studied across time. In a case-control study, a number of cases (e.g., crashes, or some other safety measure) are identified and are matched with controls; focus is then placed on the differences between the cases and controls. Both cohort studies and case-control studies can be useful in assessing safety.

\textbf{Fatigue measurement}

The reason why little has been done about drowsy driving is there is no simple way to objectively measure fatigue driving (Dement \protect\hyperlink{ref-dement1997perils}{1997}).

\hypertarget{driver-characteristics}{%
\subsection{Driver characteristics}\label{driver-characteristics}}

A study by Pack et al. (\protect\hyperlink{ref-pack1995characteristics}{1995}) revealed that the drivers were 25 years of age or younger in over a half of the 5,000 highway crashes in which they fell asleep whiling driving.

(Duke, Guest, and Boggess \protect\hyperlink{ref-duke2010age}{2010}) Age-related safety.

Another driver-related risk factor of driving safety critical events is drivers' age. It is reported that younger drivers will experience more crashes than older drivers (Cantor et al. \protect\hyperlink{ref-cantor2010driver}{2010}).

On the other hand, older drivers can also experience an increased risk of crash. To meet the huge demand services and supply chain management, it is very common to extend the retirement age or reemploy retired workers in developing countries (Popkin et al. \protect\hyperlink{ref-popkin2008age}{2008}). Aging drivers increase the chance of the safety critical events in three aspects: impaired eyesight, prolonged reaction time to exogenous stimuli, and vulnerability to fatigue (Di Milia et al. \protect\hyperlink{ref-di2011demographic}{2011}). Aged drivers are associated with eyesight diseases or functionality impairment, such as cataracts, narrowed peripheral vision and decreasing visual acuity (Di Milia et al. \protect\hyperlink{ref-di2011demographic}{2011}). In addition, working for truck companies often means irregular shifts and taking the night schedules, which disrupt the circadian time-keeping systems, especially for the aged workers (Moneta et al. \protect\hyperlink{ref-moneta1996time}{1996}).

Aged drivers may find it much more difficult to adjust for the sleep-wake cycle to keep pace with the schedule required by the employer company. Therefore, this disruption of the circadian systems, in turn, increases the chances to feel sleepy or fatigue for workers. It is indicated by research that the ``critical age'' of shiftwork intolerance is about 45 to 50 years, at which sleep disorder, persisting fatigue and digestive problems become the most obvious (Di Milia et al. \protect\hyperlink{ref-di2011demographic}{2011}). Young drivers are much better in the sense of physical health and resistance to fatigue compared with aged drivers, however, they are more vulnerable regarding the experience of driving. A study conducted by Clarke suggested that young drivers (17 -- 19 years old), especially males, have significantly more accidents than other drivers during the hours of darkness, on rural curves, and rear-end shunts compared with male drivers aged 20 -25 years (Clarke et al. \protect\hyperlink{ref-clarke2006young}{2006}). The reasons for these young driver accidents were not fully explained, but could largely be attributed to inexperience.

One more risk factor that could explain driving safety critical events is drivers' gender. Gender has been suggested to be related with outcomes in medical treatment, education, sports and other fields, and there is no exception for truck drivers' safety. In the first place, women are more likely to suffer from fatigue compared with men. A study found that women in general have 1.4 times higher chance of complaining of fatigue than men (Fjell et al. \protect\hyperlink{ref-fjell2008perceived}{2008}). However, females are found to have longer sleeping hours than their male counterparts of the same race (Lauderdale et al. \protect\hyperlink{ref-lauderdale2006objectively}{2006}). In that study, it was found that the mean sleep hours for white females was 6.7 hours compared with 6.1 hours for white males, and 5.9 hours for black female compared with 5.1 hours for black males even after adjusting for socioeconomic status, lifestyle and sleep apnea (Lauderdale et al. \protect\hyperlink{ref-lauderdale2006objectively}{2006}). Gender differences are huge in terms of working conditions. Females had significantly fewer working hours per week, with 47 hours versus 52 hours per week (Rotenberg et al. \protect\hyperlink{ref-rotenberg2008gender}{2008}). In general, women tend to work fewer hours within a week but are more prone to feel fatigue and have a higher risk of traffic incidences.

\hypertarget{traffic}{%
\subsection{Traffic}\label{traffic}}

Although it is generally believed that lower \textbf{speed limits} can reduce the chances of traffic crashes, studies on speed limits and traffic fatalities showed inconsistent results (Neeley and Richardson Jr \protect\hyperlink{ref-neeley2009effect}{2009}; Korkut, Ishak, and Wolshon \protect\hyperlink{ref-korkut2010freeway}{2010}; Zhu and Srinivasan \protect\hyperlink{ref-zhu2011comprehensive}{2011}; Davis et al. \protect\hyperlink{ref-davis2015longitudinal}{2015}).

\hypertarget{weather}{%
\subsection{Weather}\label{weather}}

(Naik et al. \protect\hyperlink{ref-naik2016weather}{2016})

Weather has both direct and indirect effects on drivers' safety critical events. On one hand, the increase of ambient temperature places risks on drivers' occupational safety, and possibly leads to cognition loss, heat stroke, and impairment of wakefulness. Evidence showed that the risk of mistakes and safety critical events increase in hot weather (Kjellstrom et al. \protect\hyperlink{ref-kjellstrom2009direct}{2009}; Basagaña et al. \protect\hyperlink{ref-basagana2015high}{2015}). Leard and Roth found that for a day with temperature above 80◦F there is a 9.5\% increase in fatality rates compared with a day at 50-60 F (Leard, Roth, and others \protect\hyperlink{ref-leard2015weather}{2015}). A literature review found that 11 out of 13 studies indicated an increase in unintentional injuries associated with high temperatures (Kampe, Kovats, and Hajat \protect\hyperlink{ref-im2016impact}{2016}). On the other hand, real-time extreme weather conditions such as heavy rain, fog, storm, and snow can either impair the driver's visual capability or reduce the safety of driving on the road (Chang and Chen \protect\hyperlink{ref-chang2005data}{2005}; Al-Ghamdi \protect\hyperlink{ref-al2007experimental}{2007}; Baker and Reynolds \protect\hyperlink{ref-baker1992wind}{1992}). It is to noted that the cumulative time of driving in such extreme weather conditions could increase the chances of safety critical events. Studies that explore the association between precipitation and driving safety critical events consistently find a negative relationship. The positive linear relationship between precipitation and traffic accidents can be observed in both driver accidents and pedestrian accidents (Al-Ghamdi \protect\hyperlink{ref-al2007experimental}{2007}; Graham and Glaister \protect\hyperlink{ref-graham2003spatial}{2003}). Abdel-Aty et al.~used detector and sensor data to successfully predict more than 70\% of accidents with low visibility conditions (Abdel-Aty et al. \protect\hyperlink{ref-abdel2012real}{2012}). The common problem for the literature exploring the relationship between ambient weather and safety driving critical events is the failure to include the cumulative effect of weather conditions. Instead, they all use an indicator variable to represent whether extreme weather happened during the trip or not, which could lead to potential bias in prediction models.

\hypertarget{time-of-the-day}{%
\subsection{Time of the day}\label{time-of-the-day}}

Pack et al. (\protect\hyperlink{ref-pack1995characteristics}{1995})'s analysis reported the crashes in which the drivers fell asleep occurred primarily from mid-night to 7 a.m. and from 2 p.m. to 4 p.m.

\hypertarget{predictive-models}{%
\section{Predictive models}\label{predictive-models}}

\hypertarget{overview}{%
\subsection{Overview}\label{overview}}

The most commonly used statistical models for traffic safety are logistic regression and Poisson regression. Logistic regression is commonly used to predict crash likelihood (probability) using real-time data, for example traffic and weather at 5-minute intervals (Wang, Abdel-Aty, and Lee \protect\hyperlink{ref-wang2017safety}{2017}). In contrast, Poisson regression is used to predict the crash frequency (the number of crashes) within a time period using aggregated data such as average data traffic and precipitation. I will briefly introduce the two models and then compare the two cultures of predictive models in statistical and machine learning perspective.

The parameterization of a logistic regression is shown in Model \eqref{eq:logit}.

\begin{equation}
\begin{split}
Y_i & \sim \text{Bernoulli}(p_i) \\
\log(\frac{p_i}{1 - p_i}) & = \beta_0 + \beta_1x_1 + \beta_2x_2 + \cdots + \beta_kx_k
\label{eq:logit}
\end{split}
\end{equation}

Where \(Y_i\) is a binary variable that indicates whether an event occurred or not in the \(i\)-th observation. \(p_i\) is the mean parameter of a Bernoulli distribution, which is constrained on \([0, 1]\). The logit transformation of \(p_i\) then has the range from \(-\infty\) to \(+\infty\), which equals a linear combination of the predictors \(x_1, x_2, \cdots, x_k\) and associated parameters \(\beta_0, \beta_1, \cdots, \beta_k\).

On the other hand, the parameterization of a Poisson regression is shown in Model \eqref{eq:poisson}.
\begin{equation}
\begin{split}
Y_i^\star & \sim \text{Poisson}(\mu_i) \\
\log\mu_i & = \beta_0 + \beta_1x_1 + \beta_2x_2 + \cdots + \beta_kx_k
\label{eq:poisson}
\end{split}
\end{equation}

Where \(Y_i^\star\) is the number of events in the \(i\)-th observation, which must be a non-negative integer. \(\mu_i\) is the mean and variance of the Poisson distribution, and it must be a non-negative numeric value. The logarithm of \(\mu_i\) transforms \(\mu_i\) into the range of \((-\infty, +\infty)\), which equals a linear combination of the predictors \(x_1, x_2, \cdots, x_k\) and associated parameters \(\beta_0, \beta_1, \cdots, \beta_k\).

\textbf{Explain or to predict} Shmueli and others (\protect\hyperlink{ref-shmueli2010explain}{2010}); Breiman and others (\protect\hyperlink{ref-breiman2001statistical}{2001}).

\hypertarget{bayesian-models}{%
\subsection{Bayesian models}\label{bayesian-models}}

In contrast to traditional frequentist models that view parameters as unknown but fixed values, Bayesian models view parameters as random variables that have probability distributions (Gelman et al. \protect\hyperlink{ref-gelman2013bayesian}{2013}). Researchers have subjective prior beliefs (a probability distribution) on these parameters \(p(\theta)\) before they collect any data. After observing the data \(\mathbf{X}\), the researchers could change their prior beliefs. Therefore, the posterior distribution \(p(\theta | \mathbf{X})\) is an an unconditional distribution that is a compromise between the prior beliefs and the data. This compromise is given analytically by the Bayes Theorem (Equation \eqref{eq:bayes}).
\begin{equation}
\begin{split}
p(\theta | \mathbf{X}) & = \frac{p(\theta)p(\mathbf{X}|\theta)}{p(\mathbf{X})} \\
 & = \frac{p(\theta)p(\mathbf{X}|\theta )}{\int p(\theta)p(\mathbf{X}|\theta)d\theta}
\label{eq:bayes}
\end{split}
\end{equation}

Where \(p(\mathbf{X}|\theta)\) is the likelihood function, which reflects the data generating process that gives rise to the data observed. The denominator \(\int p(\theta)p(\mathbf{X}|\theta)d\theta\) is a normalizing constant that makes the posterior distribution integrates to one. The prior and likelihood function are straightforward since they both have analytical forms. The trickiest part of Bayesian inference is the normalizing constant in the denominator (Gelman et al. \protect\hyperlink{ref-gelman2013bayesian}{2013}; Kruschke \protect\hyperlink{ref-kruschke2014doing}{2014}).

The normalizing constant need to make the posterior distribution integrate to one since the posterior is supposed to be a probability density distribution. When there are more than two parameters in the model, the normalizing constant often becomes intractable since it involves integration in multiple dimensions. Modern Bayesian inference often uses numerical methods such as Markov chain Monte Carlo (MCMC) methods to directly sample from this posterior distribution, or the integrated Laplace approximation to approximate this constant. However, this numerical methods often fail or take an inhibitively long time to solve the problem with the presence of high-dimensional data or very tall data in this era.

There are several strengths of Bayesian models over traditional Frequentist models. First, the probabilistic distribution of parameters, posterior credibles intervals, and posterior predictive distributions account for the uncertainty in parameters and the data generating process. They also have straightforward and intuitive interpretations. Second, Bayesian models incorporate prior information \(p(\theta)\) into the statistical model, which can be useful when there is sufficient prior background information. This prior belief is particularly useful for estimation in high-dimensional, sparse data settings, and complex statistical models such as hierarchical models (Betancourt and Girolami \protect\hyperlink{ref-betancourt2015hamiltonian}{2015}). Lastly, Bayesian models are scalable to complex data generating process. This is because modern Bayesian estimation is powered by numerical methods and simulation, which in essence only requires researchers to specify the priors and likelihood function. The difficulty of written the likelihood function is minimal compared to traditional Frequentist approaches such as the maximum likelihood estimation, which scales with the complexity of models (Lambert \protect\hyperlink{ref-lambert2018student}{2018}).

\hypertarget{hierarchical-models}{%
\subsection{Hierarchical models}\label{hierarchical-models}}

Most studies on traffic safety assume that the sampling unit is a spatial-temporal segment, which is a specific section of a road with relatively high rate of crashes during a period. However, it is not sufficient to only study the occasions where the crashes are more likely to occur; we must also study the non-crashes and compare them with crashes. On the other hand, these studies that focus on road segments ignore driver-level unobserved effects. It is reported that the chance of having crashes for truck drivers with crash history in the past year is nearly twice as high as those without crash history in the past year (Cantor et al. \protect\hyperlink{ref-cantor2010driver}{2010}). Most motor carrier insurance companies and employers also view historical safety events as an important measure of the driver's performance. Therefore, it is more natural to use driver-focused models to account for unobserved variation and characteristics associated with vehicle drivers (Huang and Abdel-Aty \protect\hyperlink{ref-huang2010multilevel}{2010}).

In the Bayesian perspective, a hierarchical model is a statistical model with the probability distribution of one parameter depends on another parameter (Kruschke and Vanpaemel \protect\hyperlink{ref-kruschke2015bayesian}{2015}). Suppose we have a model with two parameters \(\alpha, \beta\) and data \(D\). The joint prior distribution of the two parameters is \(p(\alpha, \beta)\). According to the Bayes Theorem, the posterior distribution is proportional to the product of the prior distribution and the likelihood function: \(P(\alpha, \beta|D) \propto P(\alpha, \beta)P(D|\alpha,\beta)\). In a hierarchical model setting, the product can be factored as a chain of products among parameters, also known as conditional independence, such as \(P(\alpha, \beta)P(D|\alpha,\beta) = P(D|\beta)P(\beta|\alpha)P(\alpha)\). In this parameterization, the parameter \(\alpha\) is known as the hyperparameter because it gives rise to the parameter \(\beta\) (the parameter of a parameter) (Kruschke and Vanpaemel \protect\hyperlink{ref-kruschke2015bayesian}{2015}).

Model \eqref{eq:hierarchicalbayes} demonstrates a random-intercept hierarchical logitistic regression that predicts the likelihood of safety events. The outcome \(Y_{i, d(i)}\) is a binary variable that indicates whether a safety event occurred or not, and it has a Bernoulli distribution with the mean parameter \(p_{i, d(i)}\). The logit transformation of \(p_{i, d(i)}\) can then be predicted by \(k\) variables \(x_1, x_2, \cdots, x_k\). The random intercept \(\beta_{0, d(i)}\) determines that this is hierarchical model since they vary across different drivers \(d(i)\). This model assumes that these random intercepts are sampled from a population of drivers with the mean of \(\mu_0\) and standard deviation of \(\sigma_0\), which are known as hyperparameters.
\begin{equation}
\begin{split}
Y_{i, d(i)} &\sim \text{Bernoulli}(p_{i, d(i)})\\
\log\frac{p_{i, d(i)}}{1-p_{i, d(i)}} &= \beta_{0, d(i)} + \beta_1x_{1,i} + \beta_2x_{2,i} + \cdots + \beta_kx_{k,i}\\
\beta_{0, d(i)} &\sim N(\mu_0, \sigma_0^2), \quad k = 1, 2, \cdots, D
\label{eq:hierarchicalbayes}
\end{split}
\end{equation}

Compared with traditional fixed-effects models that either pool all groups of data or estimate separate models individually for each group, a hierarchical model has the advantage of partial pooling across different groups (McElreath \protect\hyperlink{ref-mcelreath2018statistical}{2018}). This partial pooling shrinks group-level parameter estimates towards the group mean and shares information across groups. Therefore, with reasonable assumptions on the data generating process, estimates from a hierarchical model are generally more robust to extreme observations and reasonably accurate for those groups with sparse data (Gelman and Hill \protect\hyperlink{ref-gelman2006data}{2006}; Lambert \protect\hyperlink{ref-lambert2018student}{2018}).

Hierarchical models also come with costs. They are particularly known for its complexity to estimate to coefficients in both Frequetist maximum likelihood and Bayesian estimation. The de facto way of current Bayesian estimation is Markov chain Monte Carlo (MCMC). However, in the hierarchical model setting, it is difficult for MCMC to efficiently sample from the posterior distributions of hyperparameters due to the correlation between different levels of parameters, as well as the large number of parameters created by the hierarchical structure.

\hypertarget{markov-chain-monte-carlo-mcmc}{%
\subsection{Markov chain Monte Carlo (MCMC)}\label{markov-chain-monte-carlo-mcmc}}

In modern statistics, Bayesian inference almost indispensably relies on Markov chain Monte Carlo (MCMC) sampling to overcome the intractable denominator in the Bayes Theroem (Equation \eqref{eq:bayes}). A \textbf{Monte Carlo simulation} is a technique to understand a target distribution by generating a large amount of random values from that distribution (Kruschke \protect\hyperlink{ref-kruschke2014doing}{2014}). A \textbf{Markov chain} has the property that the probability distribution of the observation \(i\) only depends on the previous observation \(i-1\), not on any one prior to observation \(i-1\), as demonstrated in Equation \eqref{eq:markovchain}.
\begin{equation}
P\left(X_{i}=x_{i} | X_{1}=x_{1}, X_{2}=x_{2}, \ldots, X_{i-1}=x_{i-1}\right) = P\left(X_{i}=x_{i} | X_{i-1}=x_{i-1}\right)
\label{eq:markovchain}
\end{equation}

Integrating Markov chains and Monte Carlo simulations, the MCMC method can characterize an unknown unconditional distribution without knowing its all mathematical properties by sampling from the distribution (Van Ravenzwaaij, Cassey, and Brown \protect\hyperlink{ref-van2018simple}{2018}). It has been widely applied in multiple fields such as statistics, physics, chemistry, and computer science (Craiu and Rosenthal \protect\hyperlink{ref-craiu2014bayesian}{2014}). The most notable application of MCMC is probably in Bayesian inference, in which it has been used to draw samples from the posterior distribution and calculate relevant statistics (such as mean, standard deviation, and intervals).

The first proposal of using MCMC dates to the paper by Metropolis et al. (\protect\hyperlink{ref-metropolis1953equation}{1953}), in which they tried to solve an intractable integral with a random walk MCMC. The Metropolis algorithm starts with a randomly defined initial value of the parameter \(\theta\). From a pre-defined symmetric proposal probability distribution \(p(\theta | \mathbf{x})\), it then draw a proposal parameter value \(\theta^{(\text{prop})}\), which only depends on the current parameter value \(\theta^{(t)}\). This proposal value will be accepted with the probability of \(\alpha\) defined in Equation \eqref{eq:metropolisalpha}.
\begin{equation}
\alpha = \min\bigg(1, \frac{p(\theta^{(\text{prop})}|\mathbf{x})}{p(\theta^{(t)}|\mathbf{x})}\bigg)
\label{eq:metropolisalpha}
\end{equation}

This proposal and acceptance with probability steps will be iterated for a pre-define number of times. When the Metropolis algorithm reaches a steady state, these proposal values are random values drawn from the posterior distribution of parameter \(\theta\), which can be used to describe and characterize the posterior distribution.

After decades of successful empirical trials in physics, Hastings (\protect\hyperlink{ref-hastings1970monte}{1970}) proposed a more generalized form of the Metropolis algorithm, in which the proposal distribution can be arbitrary, but the acceptance probability \(\alpha^\star\) is modified as shown in Equation \eqref{eq:MHalpha}. This Metropolis-Hasting (MH) algorithm is the most classic and widely-known MCMC algorithm used in multiple fields.

Let \(p(\mathbf{\theta|X})\) be the posterior distribution we want to know, then the \emph{Metropolis-Hasting algorithm} is:

\begin{enumerate}
\def\labelenumi{\arabic{enumi}.}
\tightlist
\item
  Let \(\theta^{(1)}\) denote an initial value for the continuous state Markov chain,
\item
  Set \(t = 1\),
\item
  Let \(q\) be the proposal density which can depend on the current state \(\theta^{(t)}\). Simulate one observation \(\theta^{(\text{prop})}\) from \(q(\theta^{(\text{prop})} | \theta^{(t)})\),
\item
  Compute the following probability:
  \begin{equation}
  \alpha^\star = \min\bigg(1, \frac{p\left(\theta^{(\text{prop})} | \boldsymbol{x}\right)}{p\left(\theta^{(t)} | \boldsymbol{x}\right)} \frac{q\left(\theta^{(t)} | \theta^{(\text{prop})}\right)}{q\left(\theta^{(\text{prop})} | \theta^{(t)}\right)}\bigg)
  \label{eq:MHalpha}
  \end{equation}
\item
  Set \(\theta^{(t+1)} = \theta^{(\text{prop})}\) with the probability of \(\alpha^\star\); otherwise set \(\theta^{(t+1)} = \theta^{(t)}\). Set \(t \leftarrow t + 1\) and return to 3 until the desired number of iterations is reached.
\end{enumerate}

Although the M-H algorithm is simple and powerful for performing MCMC, its performance highly depends on the choice of the proposal distribution. When there are a few parameters in the model and the proposal distribution is not well-designed, the M-H algorithm will have a very low acceptance rate, which makes the M-H algorithm very inefficient. In view of this issue, Gibbs sampler was proposed with the idea that the proposed values are always accepted and each parameter is updated one at a time by generating samples from the conditional distributions (Geman and Geman \protect\hyperlink{ref-geman1987stochastic}{1987}; Gelfand and Smith \protect\hyperlink{ref-gelfand1990sampling}{1990}; Lambert \protect\hyperlink{ref-lambert2018student}{2018}). The development of the software \emph{Bayesian inference Using Gibbs Sampler (BUGS)} (Lunn et al. \protect\hyperlink{ref-lunn2000winbugs}{2000}, \protect\hyperlink{ref-lunn2009bugs}{2009}) was critical in increasing the popularity of applied Bayesian analyses considering its support for a wide variety of statistical distributions, automatic application of the Gibbs Sampler, and numerous textbooks, tutorials and discussion.

Suppose \(\mathbf{\theta} = [\theta_1, \theta_2, \cdots, \theta_k]\) is a \(k\)-dimensional parameter. Let \(\mathbf{X}\) denote the data. The \emph{Gibss sampling} algorithm is then:

\begin{enumerate}
\def\labelenumi{\arabic{enumi}.}
\tightlist
\item
  Begin with an estimate \(\mathbf{\theta}^{(0)} = [\theta_1^{(0)}, \theta_2^{(0)}, \cdots , \theta_k^{(0)}]\) in the parameter space,
\item
  Set \(t = 1\),
\item
  Simulate \(\theta_1^{(t)}\) from \(p(\theta_1|\theta_2^{(t-1)}, \theta_3^{(t-1)},\cdots , \theta_k^{(t-1)}, \mathbf{X})\),
\item
  Simulate \(\theta_2^{(t)}\) from \(p(\theta_2|\theta_1^{(t)}, \theta_3^{(t-1)},\cdots , \theta_k^{(t-1)}, \mathbf{X})\),
\item
  \(\cdots\),
\item
  Simulate \(\theta_k^{(t)}\) from \(p(\theta_k|\theta_1^{(t)}, \theta_3^{(t)},\cdots , \theta_{k-1}^{(t)}, \mathbf{X})\),
\item
  Set \(t \leftarrow t + 1\) and repeat steps \(3-6\) for a pre-specified number of iterations and make sure the Gibbs sampler reaches the steady state for a sufficient number of iterations.
\end{enumerate}

The generality of the M-H algorithm and Gibbs sampler and the simplicity in software packages in \texttt{R} or \texttt{BUGS} help them gain popularity among applied researchers in the recent 30 years. However, as more and more data are available in applied field, the performance of the two most popular MCMC methods has been widely criticized (Betancourt \protect\hyperlink{ref-betancourt2019convergence}{2019}). The performance of the M-H algorithm crucially depends on the proposal distribution. An efficient proposal distribution in M-H algorithm should generate random draws with less auto-correlation, which enables more effective exploration of the parameter space (Quiroz \protect\hyperlink{ref-quiroz2015bayesian}{2015}). On the other hand, the performance of the Gibss Sampler crucially depends on the parameter structure. If there is a significant correlation between parameter estimates, the Gibbs Sampler will become very inefficient as the geometry of the distribution is not aligned with the stepping directions of each sampler (Lambert \protect\hyperlink{ref-lambert2018student}{2018}).

\hypertarget{scalable-bayesian-models}{%
\section{Scalable Bayesian models}\label{scalable-bayesian-models}}

Recent ten years witnessed an explosive growth of data size and dimensionality. This poses a major challenge to Bayesian methods using MCMC. Traditional MCMC algorithm need to evaluate the entire data at each step of iteration, which could be expensive for computation in the case of tall data (Bardenet, Doucet, and Holmes \protect\hyperlink{ref-bardenet2017markov}{2017}). In applied analysis, researchers often need to set thousands of iterations to reach stable posterior distribution, which takes hours or days to implement a single model. Besides, when the researchers have high dimensional data where high-probability regions are concentrated on a extremely limited region of sample space, it would very hard for random-walk MCMC to generate samples from these small regions (Barp et al. \protect\hyperlink{ref-barp2018geometry}{2018}). Hierarchical models even complicate this issue by adding random parameters for each subgroup, which further grows the dimensionality of parameter space. Furthermore, when there is high correlation between different parameters that often occur in the case of many parameters, neither the M-H algorithm or Gibbs sampler can efficiently generate samples from the posterior distribtion. All the aforementioned problems motivate researchers in different fields to develop different scalable algorithms to make Bayesian inference for big data.

\hypertarget{hamiltonian-monte-carlo-hmc}{%
\subsection{Hamiltonian Monte Carlo (HMC)}\label{hamiltonian-monte-carlo-hmc}}

The M-H algorithm and Gibbs sampler can be very inefficient in big data settings because of sparse high-density parameter space, high costs of evaluating the entire data at each step, or a high correlation between parameters. Originally proposed by Duane et al. (\protect\hyperlink{ref-duane1987hybrid}{1987}) with the name of Hybrid Monte Carlo, the Hamiltonian Monte Carlo (HMC) modifies the random-walk behavior in M-H algorithm into a deterministic one by adding auxiliary momentum parameters \(p_n\), thus more efficiently explores the high-density regions in big data settings compared to the traditional M-H algorithm or the Gibbs sampler (Betancourt \protect\hyperlink{ref-betancourt2017conceptual}{2017}; Wang, Broccardo, and Song \protect\hyperlink{ref-wang2019hamiltonian}{2019}). Although HMC was originally proposed in 1987 (Duane et al. \protect\hyperlink{ref-duane1987hybrid}{1987}), it is only widely adopted by applied researchers in the recent five years, thanks to the development of the No-U-Turn Sampler (NUTS) (Hoffman and Gelman \protect\hyperlink{ref-hoffman2014no}{2014}) and the statistical programming language \texttt{Stan} (Carpenter et al. \protect\hyperlink{ref-carpenter2017stan}{2017}).

Let \(\boldsymbol{q}\) denote the position vector and \(\boldsymbol{p}\) denote the momentum vector in the conservative dynamics physics system. Note that \(\boldsymbol{q}\) and \(\boldsymbol{q}\) must have the same length. The combination \((\boldsymbol{q}, \boldsymbol{p})\) then defines a position-momentum phase space, which can be calculated using the conditional distribution (Neal and others \protect\hyperlink{ref-neal2011mcmc}{2011}; Betancourt \protect\hyperlink{ref-betancourt2017conceptual}{2017}):
\[\pi(\boldsymbol{p}, \boldsymbol{q}) = \pi(\boldsymbol{p}|\boldsymbol{q})\pi(\boldsymbol{q})\]

This joint distribution can also be defined in terms of the \emph{Hamiltonian}:
\[\pi(\boldsymbol{p}, \boldsymbol{q}) = e^{-H(\boldsymbol{p}, \boldsymbol{q})}\]
After a little bit of transformation, we have:
\begin{equation}
\begin{aligned}
H(\boldsymbol{p}, \boldsymbol{q}) &= -\log \pi(\boldsymbol{p}, \boldsymbol{q})\\
&= -\log\pi(\boldsymbol{p}|\boldsymbol{q}) - \log\pi(\boldsymbol{q})\\
&= K(\boldsymbol{p}, \boldsymbol{q}) + V(\boldsymbol{q})
\label{eq:hamiltonian}
\end{aligned}
\end{equation}

In the perspective of physics, the \emph{Hamiltonian} \(H(\boldsymbol{p}, \boldsymbol{q})\) is the total energy of the system, which composes of two parts: \emph{kinetic energy} \(K(\boldsymbol{p}, \boldsymbol{q})\) and \emph{potential energy} \(V(\boldsymbol{q})\). Note that the potential energy \(V(\boldsymbol{q}) = -\log\pi(\boldsymbol{q})\) is essentially the negative log of the posterior distribution of the parameter posterior density \(\boldsymbol{q}\).

In a static system, the Hamiltonian is a constant. The evolution of this system is governed by the \emph{Hamiltonian equations}:
\begin{equation}
\begin{aligned}
\frac{d \boldsymbol{q}}{dt} &= \frac{\partial H}{\boldsymbol{p}} = \frac{\partial K}{\boldsymbol{p}}\\
\frac{d \boldsymbol{p}}{dt} &= -\frac{\partial H}{\boldsymbol{q}} = -\frac{\partial K}{\boldsymbol{q}} - \frac{\partial V}{\boldsymbol{q}}
\label{eq:hamiltonianequation}
\end{aligned}
\end{equation}

It turns out that we can randomly generate high density proposals in the parameters space by taking advantage of the Hamiltonian system. Here is the general idea if the \emph{HMC algorithm} (Lambert \protect\hyperlink{ref-lambert2018student}{2018}):

\begin{enumerate}
\def\labelenumi{\arabic{enumi}.}
\tightlist
\item
  Let \(\theta^{(0)}\) denote a random initial value from a proposal distribution,
\item
  Set \(t = 1\),
\item
  Generate a random initial momentum \(m\) from a proposal distribution (typically a multivariate normal distribution),
\item
  Use the leapfrog algorithm to solve the trajectory moving over the high-density posterior parameter space under the Hamiltonian mechanism for a period,
\item
  Calculate the new momentum \(m^\prime\) and new position \(\theta^{(\text{prop})}\)
\item
  Compute the following probability:
  \begin{equation}
  \alpha^H = \min\bigg(1, \frac{p\left(\theta^{(\text{prop})} | \boldsymbol{x}\right)p(\theta^{(\text{prop})})}{p\left(\theta^{(t)} | \boldsymbol{x}\right)p(\theta^{(t)})} \frac{q(m^\prime)}{q(m)}\bigg)
  \label{eq:hmc}
  \end{equation}
\item
  Set \(\theta^{(t+1)} = \theta^{(\text{prop})}\) with the probability of \(\alpha^H\); otherwise set \(\theta^{(t+1)} = \theta^{(t)}\). Set \(t \leftarrow t + 1\) and return to 3 until the desired number of iterations is reached.
\end{enumerate}

The HMC is essentially an improved form of M-H algorithm by using the Hamiltonian to generate effective proposals instead of naive random-walk and revised form of the acceptance probability (Equation \eqref{eq:hmc}).

Two parameters need to be tuned when implementing the HMC: step size \(\epsilon\) and the optimal trajectory length \(T\). The optimal trajectory length is the product of the number of steps \(L\) and step size \(\epsilon\) (Neal and others \protect\hyperlink{ref-neal2011mcmc}{2011}; Monnahan, Thorson, and Branch \protect\hyperlink{ref-monnahan2017faster}{2017}). The step size \(\epsilon\) decides how similarly the sympletic methods (typically the leapfrog algorithm) imitates the true unnormalized posterior density. If \(\epsilon\) is too small, it will take a lot of steps for the leapfrog algorithm to explore the posterior space. If \(\epsilon\) is too big, the leapfrog algorithm will loop around and return to a place near its original step. The trajectory length \(T = \epsilon L\), which need to be tuned in similar style with \(\epsilon\): if \(L\) is too short, it will be hard to simulate distant proposal and the algorithm is inefficient; if \(L\) is too long, the trajectory will loop back and become computationally inefficient. Hand tuning these two parameters was the major obstacle to implement HMC for applied researchers.

The No-U-Turn Sampler (NUTS) proposed by Hoffman and Gelman (\protect\hyperlink{ref-hoffman2014no}{2014}) solves the difficulty of hand tuning \(\epsilon\) and \(T\) in static HMC. NUTS calculates the optimal step size \(\epsilon\) and number of steps \(L\) through a tree building algorithm (Monnahan, Thorson, and Branch \protect\hyperlink{ref-monnahan2017faster}{2017}). The tree depth \(k\) is defined as the number of doublings, resulting in \(2^k\) leapfrog steps to build the trajectory. This \(k\) is then decided by repeating the doubling iterations until the trajectory `makes a U-turn' (loops back) or diverges (the Hamiltonian expands to infinity). Therefore, the NUTS can automatically create trajectories that can efficiently explore the high-density parameter space without having to hand tune \(\epsilon\) and \(T\).

\hypertarget{integrated-laplace-approximation-inla}{%
\subsection{Integrated Laplace Approximation (INLA)}\label{integrated-laplace-approximation-inla}}

(Rue, Martino, and Chopin \protect\hyperlink{ref-Havard2009}{2009}; Lindgren, Rue, and Lindström \protect\hyperlink{ref-Lindgren2011}{2011}; Martins et al. \protect\hyperlink{ref-Thiago2013}{2013}; De Coninck et al. \protect\hyperlink{ref-Coninck2016}{2016}; Rue et al. \protect\hyperlink{ref-Rue2017}{2017}; Verbosio et al. \protect\hyperlink{ref-Verbosio2017}{2017}; Kourounis, Fuchs, and Schenk \protect\hyperlink{ref-Kourounis2018}{2018})

\hypertarget{subsampling-mcmc}{%
\subsection{Subsampling MCMC}\label{subsampling-mcmc}}

\href{https://blog.csdn.net/u013841458/article/details/82495450}{Stochastic Gradient HMC}

(Quiroz et al., \protect\hyperlink{ref-quirozsubsampling}{n.d.}, \protect\hyperlink{ref-quiroz2016block}{2016}; Gunawan et al. \protect\hyperlink{ref-gunawan2018subsampling}{2018}; Quiroz, Tran, et al. \protect\hyperlink{ref-quiroz2018speeding}{2018}; Quiroz, Kohn, et al. \protect\hyperlink{ref-quiroz2018speeding1}{2018}; Dang et al. \protect\hyperlink{ref-dang2017hamiltonian}{2017}; Quiroz \protect\hyperlink{ref-quiroz2015bayesian}{2015})

T. Chen, Fox, and Guestrin (\protect\hyperlink{ref-chen2014stochastic}{2014})

\hypertarget{conceptual-framework}{%
\section{Conceptual framework}\label{conceptual-framework}}

Cantor et al. (\protect\hyperlink{ref-cantor2010driver}{2010}) suggested three factors that cause truck crashes: driver factors, vehicle factors (type and condition), and environmental factors.

Roshandel, Zheng, and Washington (\protect\hyperlink{ref-roshandel2015impact}{2015}) proposed five factors that affect traffic safety: (a) behavioral characteristics of the driver, e.g., impairment, fatigue, distractions; (b) vehicle --- the condition of the vehicle; (c) traffic --- the traffic conditions; (d) geometry --- geometric characteristics of the road, e.g.~curve, hill, ramps, etc.; and (e) environmental --- characteristics of the surrounding environment, such as weather conditions (rain, snow, night-time driving, etc.). Traffic conditions are the most studied of these and we focus on discussing them in this subsection.

\hypertarget{gaps-in-literature}{%
\section{Gaps in literature}\label{gaps-in-literature}}

\begin{itemize}
\tightlist
\item
  A focus on crashes instead of precursors of crashes
\item
  A focus on road segments rather than drivers
\item
  A focus on case-control comparison given the rareness of truck crashes rather than rates
\end{itemize}

\hypertarget{research-aims}{%
\section{Research aims}\label{research-aims}}

\begin{enumerate}
\def\labelenumi{\arabic{enumi}.}
\item
  \textbf{Aim1}: Explore the association between truck crashes and critical events.
\item
  \textbf{Aim2}: Predict critical events using hierarchical statistical models.
\item
  \textbf{Aim3}: scalable hierarchical Bayesian models using subsampling MCMC.
\end{enumerate}

\hypertarget{methods}{%
\chapter{METHODS}\label{methods}}

\hypertarget{data-source}{%
\section{Data source}\label{data-source}}

\hypertarget{real-time-ping}{%
\subsection{Real-time ping}\label{real-time-ping}}

The J.B. Hunt Transport Services, a trucking and transportation company in the United States,
provided real-time ping data on 496 truck drivers who conducted regional work (REG, JBI00) from April 1st, 2015 to March 29th, 2016 to the research team.
A small device was installed in each of their truck, which will ping irregularly (typically every 5-30 minutes).
Each \textbf{ping} will collect real-time data on the vehicle number, date and time, latitude, longitude, driver identity number, and speed at that second.
In total, 13,187,289 pings were collected and provided to the research team.

For each of the truck drivers, if the ping data showed that the truck was not moving for more than 20 minutes, the ping data were separated into two different trips.
These ping data were then aggregated into different trips.
A \textbf{trip} is therefore defined as a continuous period of driving without stop.
The average length of a trip in this study is 2.31 hours with the standard deviation of 1.8 hours.

After the ping data were aggregated into trips, these trips data were then further divded into different shifts according to an eight-hour rest time for each driver.
A \textbf{shift} is defined as a long period of driving with potentially less than 8 hours' stops.
The Shift\_ID column in Table \ref{tab:tripsdata} shows different shifts, separated by an eight-hour threshold.
The average length of a shift in this study is 8.42 hours with the standard deviation of 2.45 hours.

\hypertarget{truck-crashes-and-critical-events}{%
\subsection{Truck crashes and critical events}\label{truck-crashes-and-critical-events}}

The company regularly collected real-time GPS location and time-stamped critical events data for all their trucks.
Four types of critical events were recorded in the critical events data:

\begin{enumerate}
\def\labelenumi{\arabic{enumi}.}
\tightlist
\item
  Hard brake
\item
  Headway
\item
  Rolling stability
\item
  Collision mitigation
\end{enumerate}

Once some thresholds with regard to the driver's operation were met, the sensor will be triggered and critical events will be recorded in their database.
There were 12,458 critical events created by the 496 truck drivers during the study period.

\hypertarget{driver-demographics}{%
\subsection{Driver demographics}\label{driver-demographics}}

These variables can be linked to the trips and crashes table via a common unique identifier.

\hypertarget{weather-data-from-the-darksky-api}{%
\subsection{\texorpdfstring{Weather data from the \texttt{DarkSky\ API}}{Weather data from the DarkSky API}}\label{weather-data-from-the-darksky-api}}

We retrieved weather variables from the Dark Sky API: \emph{precipitation intensity}, \emph{precipitation probability}, \emph{wind speed}, and \emph{visibility}.
The DarkSky API allows the users to query historic minute-by-minute weather data anywhere on the globe (The Dark Sky Company, LLC \protect\hyperlink{ref-darksky}{2019}).
According to the official document, the Dark Sky API is supported by a wide range of weather data sources, which are aggregated together to provide the most precise weather data possible for a given location (The Dark Sky API \protect\hyperlink{ref-darkskyds}{2019}).
Among several different weather data providers we tested, the Dark Sky API provides the most accurate and complete weather variables.

To reduce the cost of querying weather data for the 13 million historic ping data, we rounded the latitudes and longitudes coordinates to two decimal places, which were worth up to 1.1 kilometers
We also rounded the time to the nearest hour and ignore those stopping pings.
This scaled the original 13 million ping data down to 4.9 million unique latitude-longitude-date-time combinations.
We used the R package \texttt{darksky} to obtain weather variables for these reduced 4.9 million unique combinations (Rudis \protect\hyperlink{ref-hrbrmstr}{2018}).
The weather data for these combinations were then merged back to the origin data.
A minimal example of R code to retrieve weather data from the DarkSky API can be found in Appendix \ref{weatherdat}.

\hypertarget{road-geometry-data-from-the-openstreetmap}{%
\subsection{\texorpdfstring{Road geometry data from the \texttt{OpenStreetMap}}{Road geometry data from the OpenStreetMap}}\label{road-geometry-data-from-the-openstreetmap}}

We queried two road geometry variables from the \texttt{OpenStreetMap} (OSM) project: \emph{speed limit} and \emph{the number of lanes}.
The OSM data are collaboratively collected by over two million registered users via manual survey, GPS devices, aerial photography, and other open-access sources (Wikipedia contributors \protect\hyperlink{ref-wikiOSM}{2019}).
The OpenStreetMap Foundation supports a website to make the data freely available to the public under the Open Database License.

We queried the speed limit and the number of lanes by specifying a bounding box by defining a center point, as well as the width and height in meters in the \texttt{center\_bbox()} function available from the \texttt{osmar} R package (Eugster and Schlesinger \protect\hyperlink{ref-eugster2013osmar}{2013}).
We used real-time longitudes and latitudes as the center point and defined a \(100\times100\) meters box to retrieve the two variables.
If the \(100\times100\) meters box is too small to have any road geometry data, we expanded the box to \(500\times500\) and then \(1000\times1000\) to obtain geometry data.
If the OSM API returned data from multiple geometry structures, we took the mean as the returned values.
The R code to retrieve road geometry data can be found in Appendix \ref{roadgeometry}.

\hypertarget{analytical-plan-for-aim-1}{%
\section{Analytical Plan for Aim 1}\label{analytical-plan-for-aim-1}}

This part will be based on truck driver's trips data and crashes data.

\hypertarget{analytical-plan-for-aim-2}{%
\section{Analytical Plan for Aim 2}\label{analytical-plan-for-aim-2}}

In this study, we use vehicle drivers as the sampling unit and adopt hierarchical statistical models that accounts for both driver-level variation and trip-level variation. The workflow is to sample a certain number of drivers from a population of drivers, observe their driving trips or shifts for a specific period, then compare the safety events with non-events, and make conclusions on risk factors associated with these safety events.

\hypertarget{logistic-regression}{%
\subsection{Logistic regression}\label{logistic-regression}}

\begin{equation}
\begin{split}
Y_{i, d} &\sim \text{Bernoulli}(p_i)\\
\log\frac{p_{i, d}}{1-p_{i, d}} &= \beta_{0, d} + \beta_1x_1 + \beta_2x_2 + \cdots + \beta_kx_k\\
\beta_{0, d} &\sim N(\mu_0, \sigma_0^2), \quad k = 1, 2, \cdots, D
\label{eq:hierarchicallogit}
\end{split}
\end{equation}

\hypertarget{poisson-regression}{%
\subsection{Poisson regression}\label{poisson-regression}}

\hypertarget{non-homogeneous-poisson-process}{%
\subsection{Non-homogeneous Poisson process}\label{non-homogeneous-poisson-process}}

\textbf{Mean function of a point process}:

\[\Lambda(t) = E(N(t))\]

\(\Lambda(t)\) is the expected number of failures through time \(t\).

\textbf{Rate of Occurence of Failures (ROCOF)}: When \(\Lambda\) is differentiable, the ROCOF is:

\[\mu(t) = \frac{d}{dt}\Lambda(t)\]
The ROCOF can be interpreted as the instantaneous rate of change in the expected number of failures.

\textbf{Intensity function}: The intensity function of a point process is

\[\lambda(t) = \lim_{\Delta t \rightarrow 0}\frac{P(N(t, t+\Delta t] \geq 1)}{\Delta t}\]

When there is no simultaneous events, ROCOF is the same as intensity function.

\textbf{Nonhomogeneous Poisson Process (NHPP)}: The NHPP is a Poisson process whose intensity function is non-constant.

\textbf{Power law process (PLP)}: When the intensity function of a NHPP is:

\[\lambda(t) = \frac{\beta}{\theta}\bigg(\frac{t}{\theta}\bigg)^{\beta-1}\]

Where \(\beta > 0\) and \(\theta > 0\), the process is called the power law process (PLP).

Therefore, the mean function \(\Lambda(t)\) is the integral of the intensity function:

\[\Lambda(t) = \int_0^t \lambda(t)dt = \int_0^t \frac{\beta}{\theta}\bigg(\frac{t}{\theta}\bigg)^{\beta-1} = \bigg(\frac{t}{\theta}\bigg)^{\beta}\]

There are two forms of truncation in a NHPP:

\begin{enumerate}
\def\labelenumi{\arabic{enumi}.}
\tightlist
\item
  \textbf{Failure truncation}: When testing stops after a predetermined number of failures, the data are said to be failure truncated.
\item
  \textbf{Time truncation}: Data are said to be time truncated when testing stops at a predetermined time \(t\).
\end{enumerate}

\textbf{The first event}

The cumulative density function (cdf) of time to the first event is \(F(t_1)\): \(F_1(t_1) = P(T_1 \leq t_1) = 1 - S(t_1)\).

The survival function for the first event \(S_1(t_1)\) is:
\begin{align*}
S_1(t_1) & = P(T_1 > t_1) \\
 & = P(N(0, t_1) = 0) \quad N \text{is the number of events}\\
 & = e^{-\int_{0}^{t_1}\lambda_{u}du}(e^{-\int_{0}^{t_1}\lambda_{u}du})^0/0!\\
 & = e^{-\int_{0}^{t_1}\lambda_{u}du}
\end{align*}

The probability density function (pdf) of time to the first event can be calculated by taking the first order derivative of the cdf \(F_1(t_1)\):
\begin{align*}
f_1(t_1) & = \frac{d}{dt_1}F_1(t_1)\\
& = \frac{d}{dt_1}[1 - S_1(t_1)] \\
& = - \frac{d}{dt_1}S_1(t_1)\\
& = - \frac{d}{dt_1}e^{-\int_{0}^{t_1}\lambda (u)du}\\
& = -(-\lambda_{t_1})e^{-\int_{0}^{t_1}\lambda (u)du}\\
& = \lambda (t_1)e^{-\int_{0}^{t_1}\lambda (u)du}
\end{align*}

If this NHPP is a PLL, we plug in the intensity function \(\lambda(t) = (\beta / \theta)(t/\theta)^{\beta - 1}\), then we have:

\[f_1(t_1) = \frac{\beta}{\theta}(\frac{t_1}{\theta})^{\beta - 1}e^{-(\frac{t_1}{\theta})^\beta}, \quad t_1 > 0\]

This pdf is identical with the pdf of Weibull distribution, so we have:
\[T_1 \sim \text{Weibull}(\beta, \theta)\]

\textbf{The second event}

The Survival function of the second event given the first event occurred at \(t_2\) is:

\begin{align*}
S_2(t_2 | t_1) & = P(T_2 > t_2 | T_1 = t)\\
& = P(N(t_1, t_2) = 0|T_1 = t_1) \\
& = e^{-\int_{t_1}^{t_2}\lambda_{u}du}[\int_{t_1}^{t_2}\lambda_{u}du]^0/0!\\
& = e^{-\int_{t_1}^{t_2}\lambda_{u}du}
\end{align*}

The we can derive the pdf of \(t_2\) conditioned on \(t_1\)
\begin{equation}\label{t2}
\begin{aligned}
f(t_2|t_1) & = - \frac{d}{dt_2}S_2(t_2)\\
& = - \frac{d}{dt_2}e^{-\int_{t_1}^{t_2}\lambda(u)du}\\
& = \lambda(t_2)e^{-\int_{t_1}^{t_2}\lambda(u)du}\\
& = \frac{\beta}{\theta}(\frac{t_2}{\theta})^{\beta - 1}e^{-[(\frac{t_2}{\theta})^\beta - (\frac{t_1}{\theta})^\beta]}\\
& = \frac{\frac{\beta}{\theta}(\frac{t_2}{\theta})^{\beta - 1}e^{-(t_2/\theta)^\beta }}{e^{- (t_1/\theta)^\beta}}, \quad t_2 > t_1
\end{aligned}
\end{equation}

In the \emph{failure truncated case}, we know the total number of events \(n\) before the experiment starts. We can get the joint likelihood function for \(t_1 < t_2 < \cdots < t_n\) in the failure truncated case based on Equation \ref{t2}.
\begin{equation}\label{pdfn}
\begin{aligned}
f(t_1, t_2, \cdots, t_n) & = f(t_1)f(t_2|t_1)f(t_3|t_1, t_2) \cdots f(t_n|t_1, t_2, \dots, t_{n - 1}) \\
& = \lambda (t_1)e^{-\int_{0}^{t_1} \dot \lambda (u)du}\lambda (t_2)e^{-\int_{t_1}^{t_2} \dot \lambda (u)du}\cdots\lambda (t_n)e^{-\int_{t_{n-1}}^{t_n}\lambda (u)du}\\
& = \Big(\prod_{i=1}^n\lambda(t_i)\Big)e^{-\int_0^t\lambda(u)du}\\
& = \Big(\prod_{i=1}^n\frac{\beta}{\theta}(\frac{t_i}{\theta})^{\beta - 1}\Big)e^{-(t_n/\theta)^\beta}, \quad t_1 < t_2 < \cdots < t_n
\end{aligned}
\end{equation}

The log-likelihood function in the failure truncated case is therefore:
\[\log \ell = n\log\beta - n\beta\log\theta + (\beta - 1)\bigg(\sum_{i=1}^n\log t_i\bigg) - \Big(\frac{t_n}{\theta}\Big)^\beta\]

In the \emph{time truncated case}, we assume that the truncated time is \(\tau\). The derivation of \(f(t_1, t_2, \cdots, t_n|n)\) is messy in math, we directly give the conclusion here:

\[f(t_1, t_2, \cdots, t_n|n) = n!\prod_{i=1}^n\frac{\lambda(t_i)}{\Lambda(\tau)}\]

Therefore, the joint likelihood function for \(f(n, t_1, t_2, \cdots, t_n)\) is:
\begin{equation}\label{pdftau}
\begin{aligned}
f(n, t_1, t_2, \cdots, t_n) & = f(n)f(t_1, t_2, \cdots, t_n|n)\\
& = \frac{e^{-\int_0^\tau \lambda(u)du}[\int_0^\tau \lambda(u)du]^n}{n!}n!\frac{\prod_{i=1}^n\lambda(t_i)}{[\Lambda(\tau)]^n}\\
& = \Big(\prod_{i=1}^n\lambda(t_i) \Big)e^{-\int_0^\tau \lambda(u)du}\\
& = \Big(\prod_{i=1}^n\frac{\beta}{\theta}(\frac{t_i}{\theta})^{\beta - 1} \Big)e^{-(\tau/\theta)^\beta},\\ 
n & = 0, 1, 2, \cdots, \quad  0 < t_1 < t_2 < \cdots < t_n
\end{aligned}
\end{equation}

The log likelihood function \(l\) is then:
\begin{equation}\label{logtau}
\begin{aligned}
l & = \log \Bigg(\Big(\prod_{i=1}^n\frac{\beta}{\theta}(\frac{t_i}{\theta})^{\beta - 1}\Big)e^{-(\tau/\theta)^\beta}\Bigg)\\
& = \sum_{i=1}^n\log\Big(\frac{\beta}{\theta}(\frac{t_i}{\theta})^{\beta - 1}\Big) - (\frac{\tau}{\theta})^\beta\\
& = n\log\beta - n\beta\log\theta + (\beta - 1)\bigg(\sum_{i=1}^n\log t_i\bigg) - \Big(\frac{\tau}{\theta}\Big)^\beta
\end{aligned}
\end{equation}

\hypertarget{analytical-plan-for-aim-3}{%
\section{Analytical Plan for Aim 3}\label{analytical-plan-for-aim-3}}

\hypertarget{the-probable-content}{%
\chapter{THE PROBABLE CONTENT}\label{the-probable-content}}

This dissertation will be completed according to the three-paper model. The expected chapter headings are as follows:

Chapter 1: Introduction: The problem

\begin{enumerate}
\def\labelenumi{\Alph{enumi}.}
\tightlist
\item
  Epidemiology of head and neck squamous cell carcinoma and its association with second primary malignancies
\end{enumerate}

\begin{enumerate}
\def\labelenumi{\arabic{enumi}.}
\tightlist
\item
  Cancer
\item
  Head and neck squamous cell carcinoma
\item
  Definition of second primary malignancies
\item
  second primary malignancies and Head and neck squamous cell carcinoma
\end{enumerate}

\begin{enumerate}
\def\labelenumi{\Alph{enumi}.}
\setcounter{enumi}{1}
\tightlist
\item
  Risk factors for second primary malignancies
\end{enumerate}

\begin{enumerate}
\def\labelenumi{\arabic{enumi}.}
\tightlist
\item
  Tobacco use
\item
  Excessive alcohol use
\item
  Human papillomavirus infection
\end{enumerate}

\begin{enumerate}
\def\labelenumi{\Alph{enumi}.}
\setcounter{enumi}{2}
\tightlist
\item
  Existing literature and gaps that exist in the area of second primary malignancies in patients with head and neck squamous cell carcinoma
\end{enumerate}

Chapter 2: Literature review

\begin{enumerate}
\def\labelenumi{\Alph{enumi}.}
\item
  Introduction
\item
  Methods
\end{enumerate}

\begin{enumerate}
\def\labelenumi{\arabic{enumi}.}
\tightlist
\item
  Source of data and eligibility/exclusion criteria
\item
  Definitions of primary and secondary outcomes and covariates
\item
  Statistical analysis
\end{enumerate}

\begin{enumerate}
\def\labelenumi{\Alph{enumi}.}
\setcounter{enumi}{2}
\tightlist
\item
  Results
\end{enumerate}

\begin{enumerate}
\def\labelenumi{\arabic{enumi}.}
\tightlist
\item
  Description of studies included in the study
\item
  Primary outcome results
\item
  Secondary outcome results
\item
  Publication bias and study quality assessment
\end{enumerate}

\begin{enumerate}
\def\labelenumi{\Alph{enumi}.}
\setcounter{enumi}{3}
\tightlist
\item
  Discussion/Conclusions
\end{enumerate}

\begin{enumerate}
\def\labelenumi{\arabic{enumi}.}
\tightlist
\item
  Implications
\item
  Strengths and limitations
\item
  Future research
\end{enumerate}

Chapter 3: Aim 1 - Truck crashes and critical events

\begin{enumerate}
\def\labelenumi{\Alph{enumi}.}
\tightlist
\item
  Introduction
\item
  Methods
\end{enumerate}

\begin{enumerate}
\def\labelenumi{\arabic{enumi}.}
\tightlist
\item
  Source of data and eligibility/exclusion criteria
\item
  Definitions of primary and secondary outcomes and covariates
\item
  Statistical analysis
\end{enumerate}

\begin{enumerate}
\def\labelenumi{\Alph{enumi}.}
\setcounter{enumi}{2}
\tightlist
\item
  Results
\end{enumerate}

\begin{enumerate}
\def\labelenumi{\arabic{enumi}.}
\tightlist
\item
  Demographics of study population
\item
  Primary outcome results
\item
  Secondary outcome results
\end{enumerate}

\begin{enumerate}
\def\labelenumi{\Alph{enumi}.}
\setcounter{enumi}{3}
\tightlist
\item
  Discussion/Conclusions
\end{enumerate}

\begin{enumerate}
\def\labelenumi{\arabic{enumi}.}
\tightlist
\item
  Implications
\item
  Strengths and limitations
\item
  Future research
\end{enumerate}

Chapter 4: Aim 2 - Statistical models predicting truck critical events

\begin{enumerate}
\def\labelenumi{\Alph{enumi}.}
\tightlist
\item
  Introduction
\item
  Methods
\end{enumerate}

\begin{enumerate}
\def\labelenumi{\arabic{enumi}.}
\tightlist
\item
  Source of data and eligibility/exclusion criteria
\item
  Definitions of primary and secondary outcomes and covariates
\item
  Statistical analysis
\end{enumerate}

\begin{enumerate}
\def\labelenumi{\Alph{enumi}.}
\setcounter{enumi}{2}
\tightlist
\item
  Results
\end{enumerate}

\begin{enumerate}
\def\labelenumi{\arabic{enumi}.}
\tightlist
\item
  Demographics of study population
\item
  Primary outcome results
\item
  Secondary outcome results
\end{enumerate}

\begin{enumerate}
\def\labelenumi{\Alph{enumi}.}
\setcounter{enumi}{3}
\tightlist
\item
  Discussion/Conclusions
\end{enumerate}

\begin{enumerate}
\def\labelenumi{\arabic{enumi}.}
\tightlist
\item
  Implications
\item
  Strengths and limitations
\item
  Future research
\end{enumerate}

Chapter 5: Aim 3 - Subsampling Markov Chain Monte Carlo methods

Chapter 6: DISCUSSION

\begin{enumerate}
\def\labelenumi{\Alph{enumi}.}
\tightlist
\item
  Conclusion
\item
  Strengths and limitation
\item
  Future research
\end{enumerate}

\hypertarget{truck-crashes-and-critical-events-1}{%
\chapter{TRUCK CRASHES AND CRITICAL EVENTS}\label{truck-crashes-and-critical-events-1}}

\hypertarget{three-statistical-models-predicting-truck-critical-events}{%
\chapter{THREE STATISTICAL MODELS PREDICTING TRUCK CRITICAL EVENTS}\label{three-statistical-models-predicting-truck-critical-events}}

\hypertarget{hierarchical-logistic-model}{%
\section{Hierarchical logistic model}\label{hierarchical-logistic-model}}

\hypertarget{model-set-up}{%
\subsection{Model set up}\label{model-set-up}}

\hypertarget{bayesian-estimation-based-on-simulated-data}{%
\subsection{Bayesian estimation based on simulated data}\label{bayesian-estimation-based-on-simulated-data}}

\hypertarget{hierarchical-poisson-model}{%
\section{Hierarchical Poisson model}\label{hierarchical-poisson-model}}

\hypertarget{hierarchical-power-law-process}{%
\section{Hierarchical power law process}\label{hierarchical-power-law-process}}

*Mean function of a point process**:

\[\Lambda(t) = E(N(t))\]

\(\Lambda(t)\) is the expected number of failures through time \(t\).

\textbf{Rate of Occurence of Failures (ROCOF)}: When \(\Lambda\) is differentiable, the ROCOF is:

\[\mu(t) = \frac{d}{dt}\Lambda(t)\]
The ROCOF can be interpreted as the instantaneous rate of change in the expected number of failures.

\textbf{Intensity function}: The intensity function of a point process is

\[\lambda(t) = \lim_{\Delta t \rightarrow 0}\frac{P(N(t, t+\Delta t] \geq 1)}{\Delta t}\]

When there is no simultaneous events, ROCOF is the same as intensity function.

\textbf{Nonhomogeneous Poisson Process (NHPP)}: The NHPP is a Poisson process whose intensity function is non-constant.

\textbf{Power law process (PLP)}: When the intensity function of a NHPP is:

\[\lambda(t) = \frac{\beta}{\theta}\bigg(\frac{t}{\theta}\bigg)^{\beta-1}\]
Where \(\beta > 0\) and \(\theta > 0\), the process is called the power law process (PLP).

Therefore, the mean function \(\Lambda(t)\) is the integral of the intensity function:

\[\Lambda(t) = \int_0^t \lambda(t)dt = \int_0^t \frac{\beta}{\theta}\bigg(\frac{t}{\theta}\bigg)^{\beta-1} = \bigg(\frac{t}{\theta}\bigg)^{\beta}\]

\hypertarget{section}{%
\subsection{}\label{section}}

\hypertarget{bayesian-estimation-based-on-simulated-data-1}{%
\subsection{Bayesian estimation based on simulated data}\label{bayesian-estimation-based-on-simulated-data-1}}

\hypertarget{hierarchical-bayesian-models-using-subsampling-markove-chain-monte-carlo-methods}{%
\chapter{HIERARCHICAL BAYESIAN MODELS USING SUBSAMPLING MARKOVE CHAIN MONTE CARLO METHODS}\label{hierarchical-bayesian-models-using-subsampling-markove-chain-monte-carlo-methods}}

\begin{itemize}
\tightlist
\item
  Comparison with results given by frequentist models using REML (\texttt{lmer4})
\item
  Comparison with results given by Bayesian models using HMC (\texttt{rstan})
\item
  Comparison with results given Bayesian models using INLA (\texttt{INLA})
\end{itemize}

\hypertarget{discussion}{%
\chapter{DISCUSSION}\label{discussion}}

\singlespacing

\hypertarget{appendix}{%
\chapter*{APPENDIX}\label{appendix}}
\addcontentsline{toc}{chapter}{APPENDIX}

\hypertarget{dataquery}{%
\section*{Data query}\label{dataquery}}
\addcontentsline{toc}{section}{Data query}

\hypertarget{weatherdat}{%
\subsection*{Weather data}\label{weatherdat}}
\addcontentsline{toc}{subsection}{Weather data}

\begin{Shaded}
\begin{Highlighting}[]
\NormalTok{gps_sample =}\StringTok{ }
\StringTok{  }\KeywordTok{structure}\NormalTok{(}\KeywordTok{list}\NormalTok{(}
    \DataTypeTok{from_lat =} \KeywordTok{c}\NormalTok{(}\FloatTok{41.3473127}\NormalTok{, }\FloatTok{41.8189037}\NormalTok{, }\FloatTok{32.8258477}\NormalTok{, }\FloatTok{40.6776808}\NormalTok{, }
                 \FloatTok{40.2366043}\NormalTok{, }\FloatTok{41.3945561}\NormalTok{, }\FloatTok{32.6320605}\NormalTok{, }\FloatTok{40.5413856}\NormalTok{, }
                 \FloatTok{33.6287422}\NormalTok{, }\FloatTok{40.0692742}\NormalTok{, }\FloatTok{41.347986}\NormalTok{, }\FloatTok{37.7781459}\NormalTok{, }
                 \FloatTok{43.0843081}\NormalTok{, }\FloatTok{41.48026}\NormalTok{, }\FloatTok{43.495149}\NormalTok{, }\FloatTok{41.5228684}\NormalTok{, }
                 \FloatTok{41.5763081}\NormalTok{, }\FloatTok{47.6728665}\NormalTok{, }\FloatTok{41.0918361}\NormalTok{, }\FloatTok{41.1537819}\NormalTok{),}
    \DataTypeTok{from_lon =} \KeywordTok{c}\NormalTok{(}\OperatorTok{-}\FloatTok{74.2850908}\NormalTok{, }\FloatTok{-73.0835104}\NormalTok{, }\FloatTok{-97.0306677}\NormalTok{, }\FloatTok{-75.1450753}\NormalTok{,}
                 \FloatTok{-76.9367494}\NormalTok{, }\FloatTok{-72.8589916}\NormalTok{, }\FloatTok{-96.8538145}\NormalTok{, }\FloatTok{-74.8547061}\NormalTok{, }
                 \FloatTok{-113.7671634}\NormalTok{, }\FloatTok{-76.762612}\NormalTok{, }\FloatTok{-74.284785}\NormalTok{, }\FloatTok{-77.4615586}\NormalTok{, }
                 \FloatTok{-76.0977384}\NormalTok{, }\FloatTok{-73.2107541}\NormalTok{, }\FloatTok{-73.7727896}\NormalTok{, }\FloatTok{-74.0739204}\NormalTok{, }
                 \FloatTok{-88.1529175}\NormalTok{, }\FloatTok{-117.3224667}\NormalTok{, }\FloatTok{-74.1554972}\NormalTok{, }\FloatTok{-74.1887031}\NormalTok{), }
    \DataTypeTok{beg_time =} \KeywordTok{structure}\NormalTok{(}
      \KeywordTok{c}\NormalTok{(}\DecValTok{1453101738}\NormalTok{, }\DecValTok{1437508088}\NormalTok{, }\DecValTok{1436195038}\NormalTok{, }\DecValTok{1435243088}\NormalTok{, }\DecValTok{1454270680}\NormalTok{, }
        \DecValTok{1432210106}\NormalTok{, }\DecValTok{1438937772}\NormalTok{, }\DecValTok{1446486480}\NormalTok{, }\DecValTok{1450191622}\NormalTok{, }\DecValTok{1449848630}\NormalTok{, }
        \DecValTok{1457597084}\NormalTok{, }\DecValTok{1432870446}\NormalTok{, }\DecValTok{1457968284}\NormalTok{, }\DecValTok{1451298724}\NormalTok{, }\DecValTok{1431503502}\NormalTok{, }
        \DecValTok{1443416864}\NormalTok{, }\DecValTok{1438306368}\NormalTok{, }\DecValTok{1445540454}\NormalTok{, }\DecValTok{1452619392}\NormalTok{, }\DecValTok{1436091072}\NormalTok{), }
      \DataTypeTok{class =} \KeywordTok{c}\NormalTok{(}\StringTok{"POSIXct"}\NormalTok{, }\StringTok{"POSIXt"}\NormalTok{), }\DataTypeTok{tzone =} \StringTok{"UTC"}\NormalTok{)), }
    \DataTypeTok{.Names =} \KeywordTok{c}\NormalTok{(}\StringTok{"from_lat"}\NormalTok{, }\StringTok{"from_lon"}\NormalTok{, }\StringTok{"beg_time"}\NormalTok{), }
    \DataTypeTok{row.names =} \KeywordTok{c}\NormalTok{(}\OtherTok{NA}\NormalTok{, 20L), }
    \DataTypeTok{class =} \KeywordTok{c}\NormalTok{(}\StringTok{"tbl_df"}\NormalTok{, }\StringTok{"tbl"}\NormalTok{, }\StringTok{"data.frame"}\NormalTok{))}
\NormalTok{gps_sample}
\end{Highlighting}
\end{Shaded}

\begin{Shaded}
\begin{Highlighting}[]
\KeywordTok{library}\NormalTok{(darksky)}
\NormalTok{add_var =}\StringTok{ }\ControlFlowTok{function}\NormalTok{(dat)\{}
\NormalTok{  dat[,}\KeywordTok{c}\NormalTok{(}\StringTok{"time"}\NormalTok{, }\StringTok{"summary"}\NormalTok{, }\StringTok{"icon"}\NormalTok{, }\StringTok{"precipIntensity"}\NormalTok{, }
         \StringTok{"precipProbability"}\NormalTok{, }\StringTok{"temperature"}\NormalTok{, }\StringTok{"apparentTemperature"}\NormalTok{, }
         \StringTok{"dewPoint"}\NormalTok{, }\StringTok{"humidity"}\NormalTok{, }\StringTok{"pressure"}\NormalTok{, }\StringTok{"windSpeed"}\NormalTok{, }\StringTok{"windGust"}\NormalTok{, }
         \StringTok{"windBearing"}\NormalTok{, }\StringTok{"cloudCover"}\NormalTok{, }\StringTok{"visibility"}\NormalTok{)] =}\StringTok{ }\OtherTok{NA}
  \KeywordTok{return}\NormalTok{(dat)}
\NormalTok{\}}

\ControlFlowTok{for}\NormalTok{(i }\ControlFlowTok{in} \DecValTok{1}\OperatorTok{:}\KeywordTok{nrow}\NormalTok{(gps_sample))\{}
\NormalTok{  t =}\StringTok{ }\KeywordTok{get_forecast_for}\NormalTok{(gps_sample}\OperatorTok{$}\NormalTok{from_lat[i], gps_sample}\OperatorTok{$}\NormalTok{from_lon[i], }
\NormalTok{                       gps_sample}\OperatorTok{$}\NormalTok{beg_time[i])}
\NormalTok{  gps_sample}\OperatorTok{$}\NormalTok{summary[i] =}\StringTok{ }\KeywordTok{ifelse}\NormalTok{(}\KeywordTok{is.null}\NormalTok{(t[[}\DecValTok{3}\NormalTok{]]}\OperatorTok{$}\NormalTok{summary), }\OtherTok{NA}\NormalTok{, }
\NormalTok{                                 t[[}\DecValTok{3}\NormalTok{]]}\OperatorTok{$}\NormalTok{summary)}
\NormalTok{  gps_sample}\OperatorTok{$}\NormalTok{icon[i] =}\StringTok{ }\KeywordTok{ifelse}\NormalTok{(}\KeywordTok{is.null}\NormalTok{(t[[}\DecValTok{3}\NormalTok{]]}\OperatorTok{$}\NormalTok{icon), }\OtherTok{NA}\NormalTok{, t[[}\DecValTok{3}\NormalTok{]]}\OperatorTok{$}\NormalTok{icon)}
\NormalTok{  gps_sample}\OperatorTok{$}\NormalTok{precipIntensity[i] =}\StringTok{ }\KeywordTok{ifelse}\NormalTok{(}\KeywordTok{is.null}\NormalTok{(t[[}\DecValTok{3}\NormalTok{]]}\OperatorTok{$}\NormalTok{precipIntensity), }
                                         \OtherTok{NA}\NormalTok{, t[[}\DecValTok{3}\NormalTok{]]}\OperatorTok{$}\NormalTok{precipIntensity)}
\NormalTok{  gps_sample}\OperatorTok{$}\NormalTok{precipProbability[i] =}\StringTok{ }\KeywordTok{ifelse}\NormalTok{(}\KeywordTok{is.null}\NormalTok{(t[[}\DecValTok{3}\NormalTok{]]}\OperatorTok{$}\NormalTok{precipProbability), }
                                           \OtherTok{NA}\NormalTok{, t[[}\DecValTok{3}\NormalTok{]]}\OperatorTok{$}\NormalTok{precipProbability)}
\NormalTok{  gps_sample}\OperatorTok{$}\NormalTok{temperature[i] =}\StringTok{ }\KeywordTok{ifelse}\NormalTok{(}\KeywordTok{is.null}\NormalTok{(t[[}\DecValTok{3}\NormalTok{]]}\OperatorTok{$}\NormalTok{temperature), }\OtherTok{NA}\NormalTok{, }
\NormalTok{                                     t[[}\DecValTok{3}\NormalTok{]]}\OperatorTok{$}\NormalTok{temperature)}
\NormalTok{  gps_sample}\OperatorTok{$}\NormalTok{apparentTemperature[i] =}\StringTok{ }\KeywordTok{ifelse}\NormalTok{(}\KeywordTok{is.null}\NormalTok{(}
\NormalTok{    t[[}\DecValTok{3}\NormalTok{]]}\OperatorTok{$}\NormalTok{apparentTemperature), }\OtherTok{NA}\NormalTok{, t[[}\DecValTok{3}\NormalTok{]]}\OperatorTok{$}\NormalTok{apparentTemperature)}
\NormalTok{  gps_sample}\OperatorTok{$}\NormalTok{dewPoint[i] =}\StringTok{ }\KeywordTok{ifelse}\NormalTok{(}\KeywordTok{is.null}\NormalTok{(t[[}\DecValTok{3}\NormalTok{]]}\OperatorTok{$}\NormalTok{dewPoint), }\OtherTok{NA}\NormalTok{, }
\NormalTok{                                  t[[}\DecValTok{3}\NormalTok{]]}\OperatorTok{$}\NormalTok{dewPoint)}
\NormalTok{  gps_sample}\OperatorTok{$}\NormalTok{humidity[i] =}\StringTok{ }\KeywordTok{ifelse}\NormalTok{(}\KeywordTok{is.null}\NormalTok{(t[[}\DecValTok{3}\NormalTok{]]}\OperatorTok{$}\NormalTok{humidity), }\OtherTok{NA}\NormalTok{, }
\NormalTok{                                  t[[}\DecValTok{3}\NormalTok{]]}\OperatorTok{$}\NormalTok{humidity)}
\NormalTok{  gps_sample}\OperatorTok{$}\NormalTok{pressure[i] =}\StringTok{ }\KeywordTok{ifelse}\NormalTok{(}\KeywordTok{is.null}\NormalTok{(t[[}\DecValTok{3}\NormalTok{]]}\OperatorTok{$}\NormalTok{pressure), }\OtherTok{NA}\NormalTok{, }
\NormalTok{                                  t[[}\DecValTok{3}\NormalTok{]]}\OperatorTok{$}\NormalTok{pressure)}
\NormalTok{  gps_sample}\OperatorTok{$}\NormalTok{windSpeed[i] =}\StringTok{ }\KeywordTok{ifelse}\NormalTok{(}\KeywordTok{is.null}\NormalTok{(t[[}\DecValTok{3}\NormalTok{]]}\OperatorTok{$}\NormalTok{windSpeed), }\OtherTok{NA}\NormalTok{, }
\NormalTok{                                   t[[}\DecValTok{3}\NormalTok{]]}\OperatorTok{$}\NormalTok{windSpeed)}
\NormalTok{  gps_sample}\OperatorTok{$}\NormalTok{windGust[i] =}\StringTok{ }\KeywordTok{ifelse}\NormalTok{(}\KeywordTok{is.null}\NormalTok{(t[[}\DecValTok{3}\NormalTok{]]}\OperatorTok{$}\NormalTok{windGust), }\OtherTok{NA}\NormalTok{, }
\NormalTok{                                  t[[}\DecValTok{3}\NormalTok{]]}\OperatorTok{$}\NormalTok{windGust)}
\NormalTok{  gps_sample}\OperatorTok{$}\NormalTok{windBearing[i] =}\StringTok{ }\KeywordTok{ifelse}\NormalTok{(}\KeywordTok{is.null}\NormalTok{(t[[}\DecValTok{3}\NormalTok{]]}\OperatorTok{$}\NormalTok{windBearing), }\OtherTok{NA}\NormalTok{, }
\NormalTok{                                     t[[}\DecValTok{3}\NormalTok{]]}\OperatorTok{$}\NormalTok{windBearing)}
\NormalTok{  gps_sample}\OperatorTok{$}\NormalTok{cloudCover[i] =}\StringTok{ }\KeywordTok{ifelse}\NormalTok{(}\KeywordTok{is.null}\NormalTok{(t[[}\DecValTok{3}\NormalTok{]]}\OperatorTok{$}\NormalTok{cloudCover), }\OtherTok{NA}\NormalTok{, }
\NormalTok{                                    t[[}\DecValTok{3}\NormalTok{]]}\OperatorTok{$}\NormalTok{cloudCover)}
\NormalTok{  gps_sample}\OperatorTok{$}\NormalTok{visibility[i] =}\StringTok{ }\KeywordTok{ifelse}\NormalTok{(}\KeywordTok{is.null}\NormalTok{(t[[}\DecValTok{3}\NormalTok{]]}\OperatorTok{$}\NormalTok{visibility), }\OtherTok{NA}\NormalTok{, }
\NormalTok{                                    t[[}\DecValTok{3}\NormalTok{]]}\OperatorTok{$}\NormalTok{visibility)}
\NormalTok{\}}
\end{Highlighting}
\end{Shaded}

\hypertarget{roadgeometry}{%
\subsection*{Road geometry data}\label{roadgeometry}}
\addcontentsline{toc}{subsection}{Road geometry data}

\begin{Shaded}
\begin{Highlighting}[]
\NormalTok{pacman}\OperatorTok{::}\KeywordTok{p_load}\NormalTok{(osmar, stringr)}
\NormalTok{src <-}\StringTok{ }\KeywordTok{osmsource_api}\NormalTok{(}\DataTypeTok{url =} \StringTok{"https://api.openstreetmap.org/api/0.6/"}\NormalTok{)}
\NormalTok{road_data =}\StringTok{ }\ControlFlowTok{function}\NormalTok{(}\DataTypeTok{i =} \DecValTok{5}\NormalTok{, }\DataTypeTok{width =} \DecValTok{100}\NormalTok{, }\DataTypeTok{data =}\NormalTok{ df3)\{}
\NormalTok{  bb <-}\StringTok{ }\KeywordTok{center_bbox}\NormalTok{(data}\OperatorTok{$}\NormalTok{lon_short[i], data}\OperatorTok{$}\NormalTok{lat_short[i], }
\NormalTok{                    width, width)}
\NormalTok{  ua =}\StringTok{ }\KeywordTok{get_osm}\NormalTok{(bb, }\DataTypeTok{source =}\NormalTok{ src)}
\NormalTok{  ua}
\NormalTok{  road_inf <-}\StringTok{ }\KeywordTok{data.frame}\NormalTok{(ua}\OperatorTok{$}\NormalTok{ways}\OperatorTok{$}\NormalTok{tags)}
  \KeywordTok{colnames}\NormalTok{(road_inf) <-}\StringTok{ }\KeywordTok{c}\NormalTok{(}\StringTok{"ID"}\NormalTok{, }\StringTok{"Key"}\NormalTok{, }\StringTok{"Value"}\NormalTok{)}
\NormalTok{  road_inf}\OperatorTok{$}\NormalTok{Key <-}\StringTok{ }\KeywordTok{as.character}\NormalTok{(road_inf}\OperatorTok{$}\NormalTok{Key)}
\NormalTok{  road_inf}\OperatorTok{$}\NormalTok{Value <-}\StringTok{ }\KeywordTok{as.character}\NormalTok{(road_inf}\OperatorTok{$}\NormalTok{Value)}
\NormalTok{  row_speed <-}\StringTok{  }\KeywordTok{which}\NormalTok{(road_inf}\OperatorTok{$}\NormalTok{Key }\OperatorTok{==}\StringTok{ "maxspeed"}\NormalTok{, }\DataTypeTok{arr.ind=}\OtherTok{TRUE}\NormalTok{)}
\NormalTok{  row_lane <-}\StringTok{ }\KeywordTok{which}\NormalTok{(road_inf}\OperatorTok{$}\NormalTok{Key }\OperatorTok{==}\StringTok{ "lanes"}\NormalTok{, }\DataTypeTok{arr.ind=}\OtherTok{TRUE}\NormalTok{)}
  
\NormalTok{  max_speed <-}\StringTok{ }\KeywordTok{as.numeric}\NormalTok{(}\KeywordTok{str_extract}\NormalTok{(road_inf[row_speed,}\StringTok{"Value"}\NormalTok{], }
                                      \StringTok{"[[:digit:]]+"}\NormalTok{))}
\NormalTok{  num_lanes <-}\StringTok{ }\KeywordTok{as.numeric}\NormalTok{(}\KeywordTok{str_extract}\NormalTok{(road_inf[row_lane,}\StringTok{"Value"}\NormalTok{], }
                                      \StringTok{"[[:digit:]]+"}\NormalTok{))}
  \KeywordTok{return}\NormalTok{(}\KeywordTok{c}\NormalTok{(}\KeywordTok{mean}\NormalTok{(max_speed), }\KeywordTok{mean}\NormalTok{(num_lanes)))}
\NormalTok{\}}

\NormalTok{loop_data =}\StringTok{ }\ControlFlowTok{function}\NormalTok{(}\DataTypeTok{start_index =} \DecValTok{1}\NormalTok{, }\DataTypeTok{loop_length =} \DecValTok{100000}\NormalTok{)\{}
\NormalTok{  end_index =}\StringTok{ }\NormalTok{start_index }\OperatorTok{+}\StringTok{ }\NormalTok{loop_length}
\NormalTok{  out_data =}\StringTok{ }\KeywordTok{data.frame}\NormalTok{(}\KeywordTok{matrix}\NormalTok{(}\DecValTok{0}\NormalTok{, }\DataTypeTok{ncol =} \DecValTok{2}\NormalTok{, }\DataTypeTok{nrow =}\NormalTok{ loop_length))}
\NormalTok{  df_index_diff =}\StringTok{ }\NormalTok{start_index}\DecValTok{-1}
  
  \ControlFlowTok{for}\NormalTok{ (i }\ControlFlowTok{in}\NormalTok{ start_index}\OperatorTok{:}\NormalTok{end_index) \{}
\NormalTok{    out_data[i}\OperatorTok{-}\NormalTok{df_index_diff,] =}\StringTok{ }\KeywordTok{road_data}\NormalTok{(i, }\DataTypeTok{data =}\NormalTok{ df)}
    \KeywordTok{print}\NormalTok{(}\KeywordTok{paste0}\NormalTok{(end_index }\OperatorTok{-}\StringTok{ }\NormalTok{i, }\StringTok{" remained ("}\NormalTok{, }
                 \KeywordTok{round}\NormalTok{((end_index }\OperatorTok{-}\StringTok{ }\NormalTok{i)}\OperatorTok{*}\DecValTok{100}\OperatorTok{/}
\StringTok{                         }\NormalTok{(end_index}\OperatorTok{-}\NormalTok{df_index_diff), }\DecValTok{3}\NormalTok{), }\StringTok{"%)"}\NormalTok{))}
\NormalTok{  \}}
  
  \KeywordTok{return}\NormalTok{(out_data)}
\NormalTok{\}}
\end{Highlighting}
\end{Shaded}

\begin{Shaded}
\begin{Highlighting}[]
\NormalTok{df =}\StringTok{ }\NormalTok{data.table}\OperatorTok{::}\KeywordTok{fread}\NormalTok{(}\StringTok{"data/20190605_ping_compressed_3digits.csv"}\NormalTok{)}
\NormalTok{dfcontainer12 =}\StringTok{ }\KeywordTok{loop_data}\NormalTok{(}\DataTypeTok{start_index =} \DecValTok{1}\NormalTok{)}
\end{Highlighting}
\end{Shaded}

\doublespacing

\cleardoublepage

\hypertarget{bibliography}{%
\chapter*{Bibliography}\label{bibliography}}
\addcontentsline{toc}{chapter}{Bibliography}

\hypertarget{refs}{}
\leavevmode\hypertarget{ref-abdel2012real}{}%
Abdel-Aty, Mohamed A, Hany M Hassan, Mohamed Ahmed, and Ali S Al-Ghamdi. 2012. ``Real-Time Prediction of Visibility Related Crashes.'' \emph{Transportation Research Part C: Emerging Technologies} 24: 288--98.

\leavevmode\hypertarget{ref-al2007experimental}{}%
Al-Ghamdi, Ali S. 2007. ``Experimental Evaluation of Fog Warning System.'' \emph{Accident Analysis \& Prevention} 39 (6): 1065--72.

\leavevmode\hypertarget{ref-ameratunga2006road}{}%
Ameratunga, Shanthi, Martha Hijar, and Robyn Norton. 2006. ``Road-Traffic Injuries: Confronting Disparities to Address a Global-Health Problem.'' \emph{The Lancet} 367 (9521): 1533--40.

\leavevmode\hypertarget{ref-aaafoundation}{}%
American Automobile Association Foundation for Traffic Safety. 2010. ``Asleep at the Wheel: The Prevalence and Impact of Drowsy Driving.'' \url{https://www.aaafoundation.org/sites/default/files/2010DrowsyDrivingReport_1.pdf}.

\leavevmode\hypertarget{ref-anderson2017exploratory}{}%
Anderson, Jason R, Jeffrey D Ogden, William A Cunningham, and Christine Schubert-Kabban. 2017. ``An Exploratory Study of Hours of Service and Its Safety Impact on Motorists.'' \emph{Transport Policy} 53: 161--74.

\leavevmode\hypertarget{ref-azam2014comparison}{}%
Azam, Khizar, Abdul Shakoor, Riaz Akbar Shah, Afzal Khan, Shaukat Ali Shah, and Muhammad Shahid Khalil. 2014. ``Comparison of Fatigue Related Road Traffic Crashes on the National Highways and Motorways in Pakistan.'' \emph{Journal of Engineering and Applied Sciences} 33 (2).

\leavevmode\hypertarget{ref-aakerstedt1988sleepiness}{}%
Åkerstedt, Torbjörn. 1988. ``Sleepiness as a Consequence of Shift Work.'' \emph{Sleep} 11 (1): 17--34.

\leavevmode\hypertarget{ref-baker1992wind}{}%
Baker, CJ, and Sheila Reynolds. 1992. ``Wind-Induced Accidents of Road Vehicles.'' \emph{Accident Analysis \& Prevention} 24 (6): 559--75.

\leavevmode\hypertarget{ref-bardenet2017markov}{}%
Bardenet, Rémi, Arnaud Doucet, and Chris Holmes. 2017. ``On Markov Chain Monte Carlo Methods for Tall Data.'' \emph{The Journal of Machine Learning Research} 18 (1): 1515--57.

\leavevmode\hypertarget{ref-barp2018geometry}{}%
Barp, Alessandro, François-Xavier Briol, Anthony D Kennedy, and Mark Girolami. 2018. ``Geometry and Dynamics for Markov Chain Monte Carlo.'' \emph{Annual Review of Statistics and Its Application} 5: 451--71.

\leavevmode\hypertarget{ref-basagana2015high}{}%
Basagaña, Xavier, Juan Pablo Escalera-Antezana, Payam Dadvand, Òscar Llatje, Jose Barrera-Gómez, Jordi Cunillera, Mercedes Medina-Ramón, and Katherine Pérez. 2015. ``High Ambient Temperatures and Risk of Motor Vehicle Crashes in Catalonia, Spain (2000--2011): A Time-Series Analysis.'' \emph{Environmental Health Perspectives} 123 (12): 1309--16.

\leavevmode\hypertarget{ref-betancourt2017conceptual}{}%
Betancourt, Michael. 2017. ``A Conceptual Introduction to Hamiltonian Monte Carlo.'' \emph{arXiv Preprint arXiv:1701.02434}.

\leavevmode\hypertarget{ref-betancourt2019convergence}{}%
---------. 2019. ``The Convergence of Markov Chain Monte Carlo Methods: From the Metropolis Method to Hamiltonian Monte Carlo.'' \emph{Annalen Der Physik} 531 (3): 1700214.

\leavevmode\hypertarget{ref-betancourt2015hamiltonian}{}%
Betancourt, Michael, and Mark Girolami. 2015. ``Hamiltonian Monte Carlo for Hierarchical Models.'' \emph{Current Trends in Bayesian Methodology with Applications} 79: 30.

\leavevmode\hypertarget{ref-braver1997tractor}{}%
Braver, Elisa R, Paul L Zador, Denise Thum, Eric L Mitter, Herbert M Baum, and Frank J Vilardo. 1997. ``Tractor-Trailer Crashes in Indiana: A Case-Control Study of the Role of Truck Configuration.'' \emph{Accident Analysis \& Prevention} 29 (1): 79--96.

\leavevmode\hypertarget{ref-breiman2001statistical}{}%
Breiman, Leo, and others. 2001. ``Statistical Modeling: The Two Cultures (with Comments and a Rejoinder by the Author).'' \emph{Statistical Science} 16 (3): 199--231.

\leavevmode\hypertarget{ref-canani2005prevalence}{}%
Canani, SF, AB John, MG Raymundi, S Schönwald, and SS Menna Barreto. 2005. ``Prevalence of Sleepiness in a Group of Brazilian Lorry Drivers.'' \emph{Public Health} 119 (10): 925--29.

\leavevmode\hypertarget{ref-cantor2010driver}{}%
Cantor, David E, Thomas M Corsi, Curtis M Grimm, and Koray Özpolat. 2010. ``A Driver Focused Truck Crash Prediction Model.'' \emph{Transportation Research Part E: Logistics and Transportation Review} 46 (5): 683--92.

\leavevmode\hypertarget{ref-carpenter2017stan}{}%
Carpenter, Bob, Andrew Gelman, Matthew D Hoffman, Daniel Lee, Ben Goodrich, Michael Betancourt, Marcus Brubaker, Jiqiang Guo, Peter Li, and Allen Riddell. 2017. ``Stan: A Probabilistic Programming Language.'' \emph{Journal of Statistical Software} 76 (1).

\leavevmode\hypertarget{ref-chang2005data}{}%
Chang, Li-Yen, and Wen-Chieh Chen. 2005. ``Data Mining of Tree-Based Models to Analyze Freeway Accident Frequency.'' \emph{Journal of Safety Research} 36 (4): 365--75.

\leavevmode\hypertarget{ref-chen2014modeling}{}%
Chen, Chen, and Yuanchang Xie. 2014. ``Modeling the Safety Impacts of Driving Hours and Rest Breaks on Truck Drivers Considering Time-Dependent Covariates.'' \emph{Journal of Safety Research} 51: 57--63.

\leavevmode\hypertarget{ref-chen2014stochastic}{}%
Chen, Tianqi, Emily Fox, and Carlos Guestrin. 2014. ``Stochastic Gradient Hamiltonian Monte Carlo.'' In \emph{International Conference on Machine Learning}, 1683--91.

\leavevmode\hypertarget{ref-clarke2006young}{}%
Clarke, David D, Patrick Ward, Craig Bartle, and Wendy Truman. 2006. ``Young Driver Accidents in the Uk: The Influence of Age, Experience, and Time of Day.'' \emph{Accident Analysis \& Prevention} 38 (5): 871--78.

\leavevmode\hypertarget{ref-craiu2014bayesian}{}%
Craiu, Radu V, and Jeffrey S Rosenthal. 2014. ``Bayesian Computation via Markov Chain Monte Carlo.'' \emph{Annual Review of Statistics and Its Application} 1: 179--201.

\leavevmode\hypertarget{ref-dalal2013economics}{}%
Dalal, Koustuv, Zhiquin Lin, Mervyn Gifford, and Leif Svanström. 2013. ``Economics of Global Burden of Road Traffic Injuries and Their Relationship with Health System Variables.'' \emph{International Journal of Preventive Medicine} 4 (12): 1442.

\leavevmode\hypertarget{ref-dang2017hamiltonian}{}%
Dang, Khue-Dung, Matias Quiroz, Robert Kohn, Minh-Ngoc Tran, and Mattias Villani. 2017. ``Hamiltonian Monte Carlo with Energy Conserving Subsampling.'' \emph{arXiv Preprint arXiv:1708.00955}.

\leavevmode\hypertarget{ref-davis2015longitudinal}{}%
Davis, Amelia, Elizabeth Hacker, Peter T Savolainen, and Timothy J Gates. 2015. ``Longitudinal Analysis of Rural Interstate Fatalities in Relation to Speed Limit Policies.'' \emph{Transportation Research Record} 2514 (1): 21--31.

\leavevmode\hypertarget{ref-Coninck2016}{}%
De Coninck, Arne, Bernard De Baets, Drosos Kourounis, Fabio Verbosio, Olaf Schenk, Steven Maenhout, and Jan Fostier. 2016. ``Needles: Toward Large-Scale Genomic Prediction with Marker-by-Environment Interaction.'' \emph{Genetics} 203 (1): 543--55. \url{https://doi.org/10.1534/genetics.115.179887}.

\leavevmode\hypertarget{ref-dement1997perils}{}%
Dement, William C. 1997. ``The Perils of Drowsy Driving.'' \emph{The New England Journal of Medicine} 337 (11): 783--84.

\leavevmode\hypertarget{ref-utah2019}{}%
Department of Transportation, Utah. 2019. ``TRUCKS Need More Time to Stop.'' \url{https://www.udot.utah.gov/trucksmart/motorist-home/stopping-distances/}.

\leavevmode\hypertarget{ref-di2011demographic}{}%
Di Milia, Lee, Michael H Smolensky, Giovanni Costa, Heidi D Howarth, Maurice M Ohayon, and Pierre Philip. 2011. ``Demographic Factors, Fatigue, and Driving Accidents: An Examination of the Published Literature.'' \emph{Accident Analysis \& Prevention} 43 (2): 516--32.

\leavevmode\hypertarget{ref-dingus2006development}{}%
Dingus, Thomas A, Vicki L Neale, Sheila G Klauer, Andrew D Petersen, and Robert J Carroll. 2006. ``The Development of a Naturalistic Data Collection System to Perform Critical Incident Analysis: An Investigation of Safety and Fatigue Issues in Long-Haul Trucking.'' \emph{Accident Analysis \& Prevention} 38 (6): 1127--36.

\leavevmode\hypertarget{ref-duane1987hybrid}{}%
Duane, Simon, Anthony D Kennedy, Brian J Pendleton, and Duncan Roweth. 1987. ``Hybrid Monte Carlo.'' \emph{Physics Letters B} 195 (2): 216--22.

\leavevmode\hypertarget{ref-duke2010age}{}%
Duke, Janine, Maya Guest, and May Boggess. 2010. ``Age-Related Safety in Professional Heavy Vehicle Drivers: A Literature Review.'' \emph{Accident Analysis \& Prevention} 42 (2): 364--71.

\leavevmode\hypertarget{ref-eugster2013osmar}{}%
Eugster, Manuel JA, and Thomas Schlesinger. 2013. ``Osmar: OpenStreetMap and R.'' \emph{The R Journal} 5 (1): 53--63.

\leavevmode\hypertarget{ref-evans2014traffic}{}%
Evans, Leonard. 2014. ``Traffic Fatality Reductions: United States Compared with 25 Other Countries.'' \emph{American Journal of Public Health} 104 (8): 1501--7.

\leavevmode\hypertarget{ref-hos2017}{}%
Federal Motor Carrier Safety Administration. 2017. ``Summary of Hours of Service Regulations.'' \url{https://www.fmcsa.dot.gov/regulations/hours-service/summary-hours-service-regulations}.

\leavevmode\hypertarget{ref-fjell2008perceived}{}%
Fjell, Ylva, Kristina Alexanderson, Mikael Nordenmark, and Carina Bildt. 2008. ``Perceived Physical Strain in Paid and Unpaid Work and the Work-Home Interface: The Associations with Musculoskeletal Pain and Fatigue Among Public Employees.'' \emph{Women \& Health} 47 (1): 21--44.

\leavevmode\hypertarget{ref-fmcsafacts2016}{}%
FMCSA. 2016. ``Fatal occupational injuries by event, 2016.'' \url{https://www.fmcsa.dot.gov/sites/fmcsa.dot.gov/files/docs/safety/data-and-statistics/84856/cmvtrafficsafetyfactsheet2016-2017.pdf}.

\leavevmode\hypertarget{ref-fmcsareport2016}{}%
---------. 2018. ``Large Truck and Bus Crash Facts 2016.'' \url{https://www.fmcsa.dot.gov/sites/fmcsa.dot.gov/files/docs/safety/data-and-statistics/398686/ltbcf-2016-final-508c-may-2018.pdf}.

\leavevmode\hypertarget{ref-gelfand1990sampling}{}%
Gelfand, Alan E, and Adrian FM Smith. 1990. ``Sampling-Based Approaches to Calculating Marginal Densities.'' \emph{Journal of the American Statistical Association} 85 (410): 398--409.

\leavevmode\hypertarget{ref-gelman2006data}{}%
Gelman, Andrew, and Jennifer Hill. 2006. \emph{Data Analysis Using Regression and Multilevel/Hierarchical Models}. Cambridge university press.

\leavevmode\hypertarget{ref-gelman2013bayesian}{}%
Gelman, Andrew, Hal S Stern, John B Carlin, David B Dunson, Aki Vehtari, and Donald B Rubin. 2013. \emph{Bayesian Data Analysis}. Chapman; Hall/CRC.

\leavevmode\hypertarget{ref-geman1987stochastic}{}%
Geman, Stuart, and Donald Geman. 1987. ``Stochastic Relaxation, Gibbs Distributions, and the Bayesian Restoration of Images.'' In \emph{Readings in Computer Vision}, 564--84. Elsevier.

\leavevmode\hypertarget{ref-goonewardene2010road}{}%
Goonewardene, Sanchia S, Khalid Baloch, Keith Porter, Ian Sargeant, and Gamini Punchihewa. 2010. ``Road Traffic Collisions---Case Fatality Rate, Crash Injury Rate, and Number of Motor Vehicles: Time Trends Between a Developed and Developing Country.'' \emph{The American Surgeon} 76 (9): 977--81.

\leavevmode\hypertarget{ref-graham2003spatial}{}%
Graham, Daniel J, and Stephen Glaister. 2003. ``Spatial Variation in Road Pedestrian Casualties: The Role of Urban Scale, Density and Land-Use Mix.'' \emph{Urban Studies} 40 (8): 1591--1607.

\leavevmode\hypertarget{ref-grimes2005compared}{}%
Grimes, David A, and Kenneth F Schulz. 2005. ``Compared to What? Finding Controls for Case-Control Studies.'' \emph{The Lancet} 365 (9468): 1429--33.

\leavevmode\hypertarget{ref-gunawan2018subsampling}{}%
Gunawan, David, Khue-Dung Dang, Matias Quiroz, Robert Kohn, and Minh-Ngoc Tran. 2018. ``Subsampling Sequential Monte Carlo for Static Bayesian Models.'' \emph{arXiv Preprint arXiv:1805.03317}.

\leavevmode\hypertarget{ref-hastings1970monte}{}%
Hastings, W Keith. 1970. ``Monte Carlo Sampling Methods Using Markov Chains and Their Applications.''

\leavevmode\hypertarget{ref-hoffman2014no}{}%
Hoffman, Matthew D, and Andrew Gelman. 2014. ``The No-U-Turn Sampler: Adaptively Setting Path Lengths in Hamiltonian Monte Carlo.'' \emph{Journal of Machine Learning Research} 15 (1): 1593--1623.

\leavevmode\hypertarget{ref-huang2010multilevel}{}%
Huang, Helai, and Mohamed Abdel-Aty. 2010. ``Multilevel Data and Bayesian Analysis in Traffic Safety.'' \emph{Accident Analysis \& Prevention} 42 (6): 1556--65.

\leavevmode\hypertarget{ref-huang2013development}{}%
Huang, Yueng-hsiang, Dov Zohar, Michelle M Robertson, Angela Garabet, Jin Lee, and Lauren A Murphy. 2013. ``Development and Validation of Safety Climate Scales for Lone Workers Using Truck Drivers as Exemplar.'' \emph{Transportation Research Part F: Traffic Psychology and Behaviour} 17: 5--19.

\leavevmode\hypertarget{ref-hyder2012addressing}{}%
Hyder, Adnan A, Katharine A Allen, Gayle Di Pietro, Claudia A Adriazola, Rochelle Sobel, Kelly Larson, and Margie Peden. 2012. ``Addressing the Implementation Gap in Global Road Safety: Exploring Features of an Effective Response and Introducing a 10-Country Program.'' \emph{American Journal of Public Health} 102 (6): 1061--7.

\leavevmode\hypertarget{ref-janakiraman2016discovery}{}%
Janakiraman, Vijay Manikandan, Bryan Matthews, and Nikunj Oza. 2016. ``Discovery of Precursors to Adverse Events Using Time Series Data.'' In \emph{Proceedings of the 2016 Siam International Conference on Data Mining}, 639--47. SIAM.

\leavevmode\hypertarget{ref-jiang2017transport}{}%
Jiang, Baoguo, Song Liang, Zhong-Ren Peng, Haozhe Cong, Morgan Levy, Qu Cheng, Tianbing Wang, and Justin V Remais. 2017. ``Transport and Public Health in China: The Road to a Healthy Future.'' \emph{The Lancet} 390 (10104): 1781--91.

\leavevmode\hypertarget{ref-im2016impact}{}%
Kampe, Eveline Otte im, Sari Kovats, and Shakoor Hajat. 2016. ``Impact of High Ambient Temperature on Unintentional Injuries in High-Income Countries: A Narrative Systematic Literature Review.'' \emph{BMJ Open} 6 (2): e010399.

\leavevmode\hypertarget{ref-kjellstrom2009direct}{}%
Kjellstrom, Tord, R Sari Kovats, Simon J Lloyd, Tom Holt, and Richard SJ Tol. 2009. ``The Direct Impact of Climate Change on Regional Labor Productivity.'' \emph{Archives of Environmental \& Occupational Health} 64 (4): 217--27.

\leavevmode\hypertarget{ref-knipling1994crashes}{}%
Knipling, Ronald R, and Jing-Shiarn Wang. 1994. \emph{Crashes and Fatalities Related to Driver Drowsiness/Fatigue}. National Highway Traffic Safety Administration Washington, DC.

\leavevmode\hypertarget{ref-korkut2010freeway}{}%
Korkut, Murat, Sherif Ishak, and Brian Wolshon. 2010. ``Freeway Truck Lane Restriction and Differential Speed Limits: Crash Analysis and Traffic Characteristics.'' \emph{Transportation Research Record} 2194 (1): 11--20.

\leavevmode\hypertarget{ref-Kourounis2018}{}%
Kourounis, D., A. Fuchs, and O. Schenk. 2018. ``Towards the Next Generation of Multiperiod Optimal Power Flow Solvers.'' \emph{IEEE Transactions on Power Systems} PP (99): 1--10. \url{https://doi.org/10.1109/TPWRS.2017.2789187}.

\leavevmode\hypertarget{ref-kruschke2014doing}{}%
Kruschke, John. 2014. \emph{Doing Bayesian Data Analysis: A Tutorial with R, Jags, and Stan}. Academic Press.

\leavevmode\hypertarget{ref-kruschke2015bayesian}{}%
Kruschke, John K, and Wolf Vanpaemel. 2015. ``Bayesian Estimation in Hierarchical Models.'' \emph{The Oxford Handbook of Computational and Mathematical Psychology}, 279--99.

\leavevmode\hypertarget{ref-kusano2012safety}{}%
Kusano, Kristofer D, and Hampton C Gabler. 2012. ``Safety Benefits of Forward Collision Warning, Brake Assist, and Autonomous Braking Systems in Rear-End Collisions.'' \emph{IEEE Transactions on Intelligent Transportation Systems} 13 (4): 1546--55.

\leavevmode\hypertarget{ref-lambert2018student}{}%
Lambert, Ben. 2018. \emph{A Student's Guide to Bayesian Statistics}. Sage.

\leavevmode\hypertarget{ref-lauderdale2006objectively}{}%
Lauderdale, Diane S, Kristen L Knutson, Lijing L Yan, Paul J Rathouz, Stephen B Hulley, Steve Sidney, and Kiang Liu. 2006. ``Objectively Measured Sleep Characteristics Among Early-Middle-Aged Adults: The Cardia Study.'' \emph{American Journal of Epidemiology} 164 (1): 5--16.

\leavevmode\hypertarget{ref-leard2015weather}{}%
Leard, Benjamin, Kevin Roth, and others. 2015. ``Weather, Traffic Accidents, and Climate Change.'' \emph{Resources for the Future Discussion Paper}, 15--19.

\leavevmode\hypertarget{ref-leechawengwongs2006role}{}%
Leechawengwongs, Manoon, Evelyn Leechawengwongs, Chakrit Sukying, and Umaporn Udomsubpayakul. 2006. ``Role of Drowsy Driving in Traffic Accidents: A Questionnaire Survey of Thai Commercial Bus/Truck Drivers.'' \emph{JOURNAL-MEDICAL ASSOCIATION OF THAILAND} 89 (11): 1845.

\leavevmode\hypertarget{ref-lemp2011analysis}{}%
Lemp, Jason D, Kara M Kockelman, and Avinash Unnikrishnan. 2011. ``Analysis of Large Truck Crash Severity Using Heteroskedastic Ordered Probit Models.'' \emph{Accident Analysis \& Prevention} 43 (1): 370--80.

\leavevmode\hypertarget{ref-Lindgren2011}{}%
Lindgren, Finn, Håvard Rue, and Johan Lindström. 2011. ``An Explicit Link Between Gaussian Fields and Gaussian Markov Random Fields: The Stochastic Partial Differential Equation Approach (with Discussion).'' \emph{Journal of the Royal Statistical Society B} 73 (4): 423--98.

\leavevmode\hypertarget{ref-litman2013transportation}{}%
Litman, Todd. 2013. ``Transportation and Public Health.'' \emph{Annual Review of Public Health} 34: 217--33.

\leavevmode\hypertarget{ref-lunn2000winbugs}{}%
Lunn, David J, Andrew Thomas, Nicky Best, and David Spiegelhalter. 2000. ``WinBUGS-a Bayesian Modelling Framework: Concepts, Structure, and Extensibility.'' \emph{Statistics and Computing} 10 (4): 325--37.

\leavevmode\hypertarget{ref-lunn2009bugs}{}%
Lunn, David, David Spiegelhalter, Andrew Thomas, and Nicky Best. 2009. ``The Bugs Project: Evolution, Critique and Future Directions.'' \emph{Statistics in Medicine} 28 (25): 3049--67.

\leavevmode\hypertarget{ref-maclean2003hazards}{}%
MacLean, Alistair W, David RT Davies, and Kris Thiele. 2003. ``The Hazards and Prevention of Driving While Sleepy.'' \emph{Sleep Medicine Reviews} 7 (6): 507--21.

\leavevmode\hypertarget{ref-Thiago2013}{}%
Martins, Thiago G., Daniel Simpson, Finn Lindgren, and Håvard Rue. 2013. ``Bayesian Computing with INLA: New Features.'' \emph{Computational Statistics and Data Analysis} 67: 68--83.

\leavevmode\hypertarget{ref-mcelreath2018statistical}{}%
McElreath, Richard. 2018. \emph{Statistical Rethinking: A Bayesian Course with Examples in R and Stan}. Chapman; Hall/CRC.

\leavevmode\hypertarget{ref-metropolis1953equation}{}%
Metropolis, Nicholas, Arianna W Rosenbluth, Marshall N Rosenbluth, Augusta H Teller, and Edward Teller. 1953. ``Equation of State Calculations by Fast Computing Machines.'' \emph{The Journal of Chemical Physics} 21 (6): 1087--92.

\leavevmode\hypertarget{ref-meuleners2015obstructive}{}%
Meuleners, Lynn, Michelle L Fraser, Matthew H Govorko, and Mark R Stevenson. 2015. ``Obstructive Sleep Apnea, Health-Related Factors, and Long Distance Heavy Vehicle Crashes in Western Australia: A Case Control Study.'' \emph{Journal of Clinical Sleep Medicine} 11 (04): 413--18.

\leavevmode\hypertarget{ref-mitler1997sleep}{}%
Mitler, Merrill M, James C Miller, Jeffrey J Lipsitz, James K Walsh, and C Dennis Wylie. 1997. ``The Sleep of Long-Haul Truck Drivers.'' \emph{New England Journal of Medicine} 337 (11): 755--62.

\leavevmode\hypertarget{ref-moneta1996time}{}%
Moneta, Giovanni B, Annette Leclerc, Jean-François Chastang, Patrick Dang Tran, and Marcel Goldberg. 1996. ``Time-Trend of Sleep Disorder in Relation to Night Work: A Study of Sequential 1-Year Prevalences Within the Gazel Cohort.'' \emph{Journal of Clinical Epidemiology} 49 (10): 1133--41.

\leavevmode\hypertarget{ref-monnahan2017faster}{}%
Monnahan, Cole C, James T Thorson, and Trevor A Branch. 2017. ``Faster Estimation of Bayesian Models in Ecology Using Hamiltonian Monte Carlo.'' \emph{Methods in Ecology and Evolution} 8 (3): 339--48.

\leavevmode\hypertarget{ref-naik2016weather}{}%
Naik, Bhaven, Li-Wei Tung, Shanshan Zhao, and Aemal J Khattak. 2016. ``Weather Impacts on Single-Vehicle Truck Crash Injury Severity.'' \emph{Journal of Safety Research} 58: 57--65.

\leavevmode\hypertarget{ref-nantulya2002neglected}{}%
Nantulya, Vinand M, and Michael R Reich. 2002. ``The Neglected Epidemic: Road Traffic Injuries in Developing Countries.'' \emph{Bmj} 324 (7346): 1139--41.

\leavevmode\hypertarget{ref-india2015}{}%
National Crime Records Bureau, Government of India. 2015. ``NCRB 2016 Report, Chapter 1A: Traffic Accidents.'' \url{http://ncrb.gov.in/StatPublications/ADSI/ADSI2015/chapter-1A\%20traffic\%20accidents.pdf}.

\leavevmode\hypertarget{ref-nsleepf}{}%
National Sleep Foundation. 2008. ``2008 State of the States Report on Drowsy Driving.'' \url{http://drowsydriving.org/resources/2008-state-of-the-states-report-on-drowsy-driving/}.

\leavevmode\hypertarget{ref-ntsb1990}{}%
National Transportation Safety Board. 1990. ``Safety Study: Fatigue, Alcohol, Other Drugs, and Medical Factors in Fatal-to-the-Driver Heavy Truck Crashes.'' National Transportation Safety Board.

\leavevmode\hypertarget{ref-neal2011mcmc}{}%
Neal, Radford M, and others. 2011. ``MCMC Using Hamiltonian Dynamics.'' \emph{Handbook of Markov Chain Monte Carlo} 2 (11): 2.

\leavevmode\hypertarget{ref-neeley2009effect}{}%
Neeley, Grant W, and Lilliard E Richardson Jr. 2009. ``The Effect of State Regulations on Truck-Crash Fatalities.'' \emph{American Journal of Public Health} 99 (3): 408--15.

\leavevmode\hypertarget{ref-nee2019road}{}%
Née, Mélanie, Benjamin Contrand, Ludivine Orriols, Cédric Gil-Jardiné, Cedric Galéra, and Emmanuel Lagarde. 2019. ``Road Safety and Distraction, Results from a Responsibility Case-Control Study Among a Sample of Road Users Interviewed at the Emergency Room.'' \emph{Accident Analysis \& Prevention} 122: 19--24.

\leavevmode\hypertarget{ref-odero2003road}{}%
Odero, Wilson, Meleckidzedeck Khayesi, and PM Heda. 2003. ``Road Traffic Injuries in Kenya: Magnitude, Causes and Status of Intervention.'' \emph{Injury Control and Safety Promotion} 10 (1-2): 53--61.

\leavevmode\hypertarget{ref-olson2016weight}{}%
Olson, Ryan, Brad Wipfli, Sharon V Thompson, Diane L Elliot, W Kent Anger, Todd Bodner, Leslie B Hammer, and Nancy A Perrin. 2016. ``Weight Control Intervention for Truck Drivers: The Shift Randomized Controlled Trial, United States.'' \emph{American Journal of Public Health} 106 (9): 1698--1706.

\leavevmode\hypertarget{ref-pack1995characteristics}{}%
Pack, Allan I, Andrew M Pack, Eric Rodgman, Andrew Cucchiara, David F Dinges, and C William Schwab. 1995. ``Characteristics of Crashes Attributed to the Driver Having Fallen Asleep.'' \emph{Accident Analysis \& Prevention} 27 (6): 769--75.

\leavevmode\hypertarget{ref-peden2004world}{}%
Peden, Margie, Richard Scurfield, David Sleet, Dinesh Mohan, Adnan A Hyder, Eva Jarawan, Colin D Mathers, and others. 2004. ``World Report on Road Traffic Injury Prevention.'' World Health Organization Geneva.

\leavevmode\hypertarget{ref-perez2005sleep}{}%
Pérez-Chada, Daniel, Alejandro J Videla, Martin E O'flaherty, Patricia Palermo, Jorgelina Meoni, Maria I Sarchi, Marina Khoury, and Joaquı́n Durán-Cantolla. 2005. ``Sleep Habits and Accident Risk Among Truck Drivers: A Cross-Sectional Study in Argentina.'' \emph{Sleep} 28 (9): 1103--8.

\leavevmode\hypertarget{ref-popkin2008age}{}%
Popkin, Stephen M, Stephanie L Morrow, Tara E Di Domenico, and Heidi D Howarth. 2008. ``Age Is More Than Just a Number: Implications for an Aging Workforce in the Us Transportation Sector.'' \emph{Applied Ergonomics} 39 (5): 542--49.

\leavevmode\hypertarget{ref-pylkkonen2015sleepiness}{}%
Pylkkönen, M, M Sihvola, HK Hyvärinen, S Puttonen, C Hublin, and M Sallinen. 2015. ``Sleepiness, Sleep, and Use of Sleepiness Countermeasures in Shift-Working Long-Haul Truck Drivers.'' \emph{Accident Analysis \& Prevention} 80: 201--10.

\leavevmode\hypertarget{ref-quiroz2015bayesian}{}%
Quiroz, Matias. 2015. ``Bayesian Inference in Large Data Problems.'' PhD thesis, Department of Statistics, Stockholm University.

\leavevmode\hypertarget{ref-quiroz2018speeding1}{}%
Quiroz, Matias, Robert Kohn, Mattias Villani, and Minh-Ngoc Tran. 2018. ``Speeding up Mcmc by Efficient Data Subsampling.'' \emph{Journal of the American Statistical Association}, 1--13.

\leavevmode\hypertarget{ref-quiroz2018speeding}{}%
Quiroz, Matias, Minh-Ngoc Tran, Mattias Villani, and Robert Kohn. 2018. ``Speeding up Mcmc by Delayed Acceptance and Data Subsampling.'' \emph{Journal of Computational and Graphical Statistics} 27 (1): 12--22.

\leavevmode\hypertarget{ref-quiroz2016block}{}%
Quiroz, Matias, Minh-Ngoc Tran, Mattias Villani, Robert Kohn, and Khue-Dung Dang. 2016. ``The Block-Poisson Estimator for Optimally Tuned Exact Subsampling Mcmc.'' \emph{arXiv Preprint arXiv:1603.08232}.

\leavevmode\hypertarget{ref-quirozsubsampling}{}%
Quiroz, Matias, Mattias Villani, Robert Kohn, Minh-Ngoc Tran, and Khue-Dung Dang. n.d. ``Subsampling Mcmc-an Introduction for the Survey Statistician.'' \emph{Sankhya A}, 1--37.

\leavevmode\hypertarget{ref-de2018near}{}%
Rome, Liz de, Julie Brown, Matthew Baldock, and Michael Fitzharris. 2018. ``Near-Miss Crashes and Other Predictors of Motorcycle Crashes: Findings from a Population-Based Survey.'' \emph{Traffic Injury Prevention} 19 (sup2): S20--S26.

\leavevmode\hypertarget{ref-roshandel2015impact}{}%
Roshandel, Saman, Zuduo Zheng, and Simon Washington. 2015. ``Impact of Real-Time Traffic Characteristics on Freeway Crash Occurrence: Systematic Review and Meta-Analysis.'' \emph{Accident Analysis \& Prevention} 79: 198--211.

\leavevmode\hypertarget{ref-rotenberg2008gender}{}%
Rotenberg, Lúcia, Luciana Fernandes Portela, Bahby Banks, Rosane Harter Griep, Frida Marina Fischer, and Paul Landsbergis. 2008. ``A Gender Approach to Work Ability and Its Relationship to Professional and Domestic Work Hours Among Nursing Personnel.'' \emph{Applied Ergonomics} 39 (5): 646--52.

\leavevmode\hypertarget{ref-hrbrmstr}{}%
Rudis, Bob. 2018. ``darksky: An R interface to the Dark Sky API.'' GitHub. \url{https://github.com/hrbrmstr/darksky}.

\leavevmode\hypertarget{ref-Havard2009}{}%
Rue, Håvard, Sara Martino, and Nicholas Chopin. 2009. ``Approximate Bayesian Inference for Latent Gaussian Models Using Integrated Nested Laplace Approximations (with Discussion).'' \emph{Journal of the Royal Statistical Society B} 71: 319--92.

\leavevmode\hypertarget{ref-Rue2017}{}%
Rue, Håvard, Andrea I. Riebler, Sigrunn H. Sørbye, Janine B. Illian, Daniel P. Simpson, and Finn K. Lindgren. 2017. ``Bayesian Computing with INLA: A Review.'' \emph{Annual Reviews of Statistics and Its Applications} 4 (March): 395--421. \url{http://arxiv.org/abs/1604.00860}.

\leavevmode\hypertarget{ref-saleh2013accident}{}%
Saleh, Joseph H, Elizabeth A Saltmarsh, Francesca M Favaro, and Loic Brevault. 2013. ``Accident Precursors, Near Misses, and Warning Signs: Critical Review and Formal Definitions Within the Framework of Discrete Event Systems.'' \emph{Reliability Engineering \& System Safety} 114: 148--54.

\leavevmode\hypertarget{ref-sallinen2005sleepiness}{}%
Sallinen, Mikael, Mikko HÄRMÄ, Pertti Mutanen, Riikka RANTA, Jussi Virkkala, and Kiti MÜLLER. 2005. ``Sleepiness in Various Shift Combinations of Irregular Shift Systems.'' \emph{Industrial Health} 43 (1): 114--22.

\leavevmode\hypertarget{ref-sedgwick2014case}{}%
Sedgwick, Philip. 2014. ``Case-Control Studies: Advantages and Disadvantages.'' \emph{Bmj} 348: f7707.

\leavevmode\hypertarget{ref-shmueli2010explain}{}%
Shmueli, Galit, and others. 2010. ``To Explain or to Predict?'' \emph{Statistical Science} 25 (3): 289--310.

\leavevmode\hypertarget{ref-solomon2004healthcare}{}%
Solomon, Andrew J, John T Doucette, Elizabeth Garland, and Thomas McGinn. 2004. ``Healthcare and the Long Haul: Long Distance Truck Drivers---a Medically Underserved Population.'' \emph{American Journal of Industrial Medicine} 46 (5): 463--71.

\leavevmode\hypertarget{ref-staton2016road}{}%
Staton, Catherine, Joao Vissoci, Enying Gong, Nicole Toomey, Rebeccah Wafula, Jihad Abdelgadir, Yi Zhou, et al. 2016. ``Road Traffic Injury Prevention Initiatives: A Systematic Review and Metasummary of Effectiveness in Low and Middle Income Countries.'' \emph{PLoS One} 11 (1): e0144971.

\leavevmode\hypertarget{ref-stern2018data}{}%
Stern, Hal S, Daniel Blower, Michael L Cohen, Charles A Czeisler, David F Dinges, Joel B Greenhouse, Feng Guo, et al. 2018. ``Data and Methods for Studying Commercial Motor Vehicle Driver Fatigue, Highway Safety and Long-Term Driver Health.'' \emph{Accident Analysis \& Prevention}.

\leavevmode\hypertarget{ref-darkskyds}{}%
The Dark Sky API. 2019. ``Data Sources.'' \url{https://darksky.net/dev/docs/sources}.

\leavevmode\hypertarget{ref-darksky}{}%
The Dark Sky Company, LLC. 2019. ``Dark Sky API --- Overview.'' \url{https://darksky.net/dev/docs}.

\leavevmode\hypertarget{ref-nsc2018}{}%
The National Safety Council. 2018. ``Vehicle Deaths Estimated at 40,000 for Third Straight Year.'' \url{https://www.nsc.org/road-safety/safety-topics/fatality-estimates}.

\leavevmode\hypertarget{ref-theofilatos2016predicting}{}%
Theofilatos, Athanasios, George Yannis, Pantelis Kopelias, and Fanis Papadimitriou. 2016. ``Predicting Road Accidents: A Rare-Events Modeling Approach.'' \emph{Transportation Research Procedia} 14: 3399--3405.

\leavevmode\hypertarget{ref-theofilatos2018impact}{}%
---------. 2018. ``Impact of Real-Time Traffic Characteristics on Crash Occurrence: Preliminary Results of the Case of Rare Events.'' \emph{Accident Analysis \& Prevention}.

\leavevmode\hypertarget{ref-bols}{}%
The United States, Bureau of Labor Statistics. 2017. ``Fatal occupational injuries by event, 2017.'' \url{https://www.bls.gov/charts/census-of-fatal-occupational-injuries/fatal-occupational-injuries-by-event-drilldown.htm}.

\leavevmode\hypertarget{ref-van2018simple}{}%
Van Ravenzwaaij, Don, Pete Cassey, and Scott D Brown. 2018. ``A Simple Introduction to Markov Chain Monte--Carlo Sampling.'' \emph{Psychonomic Bulletin \& Review} 25 (1): 143--54.

\leavevmode\hypertarget{ref-Verbosio2017}{}%
Verbosio, Fabio, Arne De Coninck, Drosos Kourounis, and Olaf Schenk. 2017. ``Enhancing the Scalability of Selected Inversion Factorization Algorithms in Genomic Prediction.'' \emph{Journal of Computational Science} 22 (Supplement C): 99--108. \url{https://doi.org/10.1016/j.jocs.2017.08.013}.

\leavevmode\hypertarget{ref-wang2017safety}{}%
Wang, Ling, Mohamed Abdel-Aty, and Jaeyoung Lee. 2017. ``Safety Analytics for Integrating Crash Frequency and Real-Time Risk Modeling for Expressways.'' \emph{Accident Analysis \& Prevention} 104: 58--64.

\leavevmode\hypertarget{ref-wang2019hamiltonian}{}%
Wang, Ziqi, Marco Broccardo, and Junho Song. 2019. ``Hamiltonian Monte Carlo Methods for Subset Simulation in Reliability Analysis.'' \emph{Structural Safety} 76: 51--67.

\leavevmode\hypertarget{ref-who2018b}{}%
WHO. 2018a. ``Road traffic injuries.'' \url{http://www.who.int/mediacentre/factsheets/fs358/en/}.

\leavevmode\hypertarget{ref-who2018}{}%
---------. 2018b. ``The Top 10 Causes of Death.'' \url{http://www.who.int/news-room/fact-sheets/detail/the-top-10-causes-of-death}.

\leavevmode\hypertarget{ref-wikiOSM}{}%
Wikipedia contributors. 2019. ``OpenStreetMap --- Wikipedia, the Free Encyclopedia.'' \url{https://en.wikipedia.org/w/index.php?title=OpenStreetMap\&oldid=900226891}.

\leavevmode\hypertarget{ref-zaloshnja2008unit}{}%
Zaloshnja, Eduard, Ted Miller, and others. 2008. ``Unit Costs of Medium and Heavy Truck Crashes.'' The United States. Federal Motor Carrier Safety Administration.

\leavevmode\hypertarget{ref-zhang2016traffic}{}%
Zhang, Guangnan, Kelvin KW Yau, Xun Zhang, and Yanyan Li. 2016. ``Traffic Accidents Involving Fatigue Driving and Their Extent of Casualties.'' \emph{Accident Analysis \& Prevention} 87: 34--42.

\leavevmode\hypertarget{ref-zhang2010road}{}%
Zhang, Wei, Omer Tsimhoni, Michael Sivak, and Michael J Flannagan. 2010. ``Road Safety in China: Analysis of Current Challenges.'' \emph{Journal of Safety Research} 41 (1): 25--30.

\leavevmode\hypertarget{ref-zhang2014study}{}%
Zhang, Xingjian, Xiaohua Zhao, Hongji Du, and Jian Rong. 2014. ``A Study on the Effects of Fatigue Driving and Drunk Driving on Drivers' Physical Characteristics.'' \emph{Traffic Injury Prevention} 15 (8): 801--8.

\leavevmode\hypertarget{ref-zhu2011comprehensive}{}%
Zhu, Xiaoyu, and Sivaramakrishnan Srinivasan. 2011. ``A Comprehensive Analysis of Factors Influencing the Injury Severity of Large-Truck Crashes.'' \emph{Accident Analysis \& Prevention} 43 (1): 49--57.

\hypertarget{vita-auctoris}{%
\chapter*{Vita Auctoris}\label{vita-auctoris}}
\addcontentsline{toc}{chapter}{Vita Auctoris}

Miao Cai was born and raised in Xinzhou district, Wuhan, Hubei Province, China.







\backmatter
\printindex


\end{document}
