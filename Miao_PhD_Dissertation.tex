% Preamble %%%%%%%%%%%%%%%%%%%%%%%%%%%%%%%%%%%%%%
\documentclass[12pt]{book}
% \makeindex

\usepackage[nottoc]{tocbibind}
\usepackage{xspace}
\usepackage[utf8]{inputenc}
\usepackage{titling}
\usepackage{sectsty}
\usepackage[T1]{fontenc}
%\usepackage[ngerman]{babel}     %deutsche Silbentrennung
\usepackage[english]{babel}     %deutsche Silbentrennung
%\usepackage[automark]{scrpage2}
\usepackage{textcomp}
\usepackage{verbatim}
\usepackage[colorlinks=true, linkcolor=blue, citecolor=blue, urlcolor=blue]{hyperref}
\usepackage{fancyhdr}
\usepackage{soul}
\usepackage{makeidx}

\usepackage{amsmath}
\usepackage{amssymb}
%\usepackage{url}
%\usepackage{cite}
\usepackage[left=2.5cm, right=2.5cm, top=2.5cm, bottom=2.5cm]{geometry}
\usepackage{graphicx}
\usepackage{url}
\usepackage{float}
\usepackage{setspace}
\usepackage{subfigure}
\usepackage{titleref}
\usepackage[round,authoryear]{natbib}
\numberwithin{equation}{chapter}
\usepackage[bf]{caption}
\usepackage{multirow}
\usepackage{longtable}
\usepackage{booktabs} % 三线表
\usepackage[figuresright]{rotating}
\usepackage{multicol} %用于实现在同一页中实现不同的分栏
\usepackage{array} % 表格居中并且能够设置宽度
% 压缩pdf体积
%\pdfcompresslevel=9
%\pdfminorversion=5
%\pdfobjcompresslevel=3


\newcommand{\tabincell}[2]{\begin{tabular}{@{}#1@{}}#2\end{tabular}}
\renewcommand{\captionfont}{\small\slshape}
\renewcommand{\contentsname}{Contents}
\newenvironment{bottompar}{\par\vspace*{\fill}}{\clearpage}% move to the end of the page
%\renewcommand{\contentsname}{TABLE OF CONTENTS}
\addto\captionsenglish{% Replace "english" with the language you use
  \renewcommand{\contentsname}%
    {TABLE OF CONTENTS}%
}
\providecommand{\tightlist}{%
  \setlength{\itemsep}{0pt}\setlength{\parskip}{0pt}}


\renewcommand{\headrulewidth}{0pt}
\pagestyle{fancy}


\begin{document}

%% -----------------------------------------------------------------------------
%% Title page
%% -----------------------------------------------------------------------------
\begin{titlepage}
\vspace{1cm}
\begin{center}
\linespread{2}\normalsize \bfseries \MakeUppercase{Predicting Truck Drivers' Critical Events: Efficient Bayesian Hierarchical Models}
\end{center}

\vspace{7cm}

\begin{center}
{Miao Cai,  M.S.}
\vspace{0.3cm}

Draft on \today

\vspace{9cm}

Dissertation Presented to the Graduate Faculty of \\
Saint Louis University in Partial Fulfillment \\
of the Requirements for the Degree of\\
Public Health Studies, Ph.D.\\
\vspace{.5cm}
\the\year
\end{center}

\end{titlepage}
\clearpage

\pagenumbering{roman}

\begin{bottompar}
\begin{center}
\textcopyright \xspace Copyright by\\
Miao Cai \\
ALL RIGHTS RESERVED\\
\vspace{.5cm}
\the\year
\end{center}
\end{bottompar}

\topskip0pt
\vspace*{\fill}
\underline{COMMITTEE IN CHARGE OF CANDIDACY:}

\vspace{.3cm}
Professor Steven E. Rigdon, Ph.D.\\
{\setlength{\parindent}{20ex}  \indent \underline{Chairperson and Advisor}}

\vspace{.3cm}
Professor Hong Xian, Ph.D.

\vspace{.3cm}
Assistant Professor Fadel Megahed, Ph.D.
\vspace*{\fill}








\hypertarget{dedication}{%
\chapter*{Dedication}\label{dedication}}
\addcontentsline{toc}{chapter}{Dedication}

To my parents.

\hypertarget{acknowledgement}{%
\chapter*{Acknowledgement}\label{acknowledgement}}
\addcontentsline{toc}{chapter}{Acknowledgement}

I want to thank my PhD mentor and committee chair Dr.~Steven E. Rigdon.

\cleardoublepage
\tableofcontents

\listoffigures
\listoftables

\mainmatter

\doublespacing

\hypertarget{introduction}{%
\chapter{INTRODUCTION}\label{introduction}}

I used the \textbf{knitr}\index{knitr} package and the \textbf{bookdown}\index{bookdown} package to compile my book. My R session information is shown below:

Package names are in bold text (e.g., \textbf{rmarkdown}), and inline code and filenames are formatted in a typewriter font (e.g., \texttt{knitr::knit(\textquotesingle{}foo.Rmd\textquotesingle{})}). Function names are followed by parentheses (e.g., \texttt{bookdown::render\_book()}).

\hypertarget{traffic-safety}{%
\section{Traffic safety}\label{traffic-safety}}

\hypertarget{truck-driver}{%
\section{Truck driver}\label{truck-driver}}

\hypertarget{literature-review}{%
\chapter{LITERATURE REVIEW}\label{literature-review}}

\hypertarget{precursors-to-crashes}{%
\section{Precursors to crashes}\label{precursors-to-crashes}}

\hypertarget{risk-factors}{%
\section{Risk factors}\label{risk-factors}}

\hypertarget{fatigue}{%
\subsection{Fatigue}\label{fatigue}}

In a recent study, \citet{stern2018Data} reviewed the research related to fatigue of commercial motor vehicle drivers. Because of the difficulty of running a controlled experiment by imposing treatments, most research designs are observational studies, that is, they compare the effects of variables that are observed, not imposed. One exception to this is a \emph{randomized encouragement design} where drivers are randomized to receive some sort of incentive to apply some treatment, but are not forced to do so. If an effect is observed, we would conclude that it is due to the incentive, not necessarily to the actual treatment. Many studies use a cohort design or a case-control study. In a cohort design, a number of drivers is identified and studied across time. In a case-control study, a number of cases (e.g., crashes, or some other safety measure) are identified and are matched with controls; focus is then placed on the differences between the cases and controls. Both cohort studies and case-control studies can be useful in assessing safety.

Among all driver-related safety critical events, fatigue has become the most pressing problem of traffic accidents. It is estimated by National Sleep Foundation that approximately 32\% of drivers in U.S drive with fatigue over twice a month {[}8{]}. Another statistic provided by American Automobile Association Foundation for Traffic Safety in 2010 said that 16.5\% of fatal traffic accidents and 12.5\% of collisions related to injuries in U.S were associated with driving with fatigue {[}9{]}. Drowsy driving is an especially common practice in less-developed countries because of cost control and tight schedule. Surveys of commercial and public road transportation companies in less-developed countries showed that employers were frequently forcing their employees to drive for longer hours and keep working even when they were exhausted {[}6, 10, 11{]}. High proportions of drowsy driving have been found in Brazilian (22\%) {[}12{]}, Argentinean (44\%) {[}13{]}, Pakistani (54\%) {[}14{]}, and Thai (75\%) {[}15{]} truck or bus drivers. The mechanism of fatigue leading to safety critical events is that a driver's capability to stay alert to ambient traffic and pedestrians will be largely impaired. The reaction time is subsequently prolonged in that situation {[}16{]}. It is estimated that 17 hours of continuous working lead to a deterioration of driving performance equivalent to a blood alcohol level of 0.05\% {[}17{]}. What makes the outcomes worse is that fatigue driving is more likely to happen on expressways and major highways where the speed limit is over 55 miles per hour {[}18{]}. This is especially concerning because fatigue driving safety critical events are more likely to result in serious injuries and fatalities, compared with non-fatigue driving safety critical events.

\hypertarget{driver-characteristics}{%
\subsection{Driver characteristics}\label{driver-characteristics}}

Another driver-related risk factor of driving safety critical events is drivers' age. In many developing countries, to meet the huge demand services and supply chain management, it is very common to extend the retirement age or reemploy retired workers {[}19{]}. Aging drivers increase the chance of the safety critical events in three aspects: impaired eyesight, prolonged reaction time to exogenous stimuli, and vulnerability to fatigue {[}20{]}. Aged drivers are associated with eyesight diseases or functionality impairment, such as cataracts, narrowed peripheral vision and decreasing visual acuity {[}20{]}. In addition, working for truck companies often means irregular shifts and taking the night schedules, which disrupt the circadian time-keeping systems, especially for the aged workers {[}21{]}.

Aged drivers may find it much more difficult to adjust for the sleep-wake cycle to keep pace with the schedule required by the employer company. Therefore, this disruption of the circadian systems, in turn, increases the chances to feel sleepy or fatigue for workers. It is indicated by research that the ``critical age'' of shiftwork intolerance is about 45 to 50 years, at which sleep disorder, persisting fatigue and digestive problems become the most obvious {[}20{]}. Young drivers are much better in the sense of physical health and resistance to fatigue compared with aged drivers, however, they are more vulnerable regarding the experience of driving. A study conducted by Clarke suggested that young drivers (17 -- 19 years old), especially males, have significantly more accidents than other drivers during the hours of darkness, on rural curves, and rear-end shunts compared with male drivers aged 20 -25 years {[}22{]}. The reasons for these young driver accidents were not fully explained, but could largely be attributed to inexperience.

One more risk factor that could explain driving safety critical events is drivers' gender. Gender has been suggested to be related with outcomes in medical treatment, education, sports and other fields, and there is no exception for truck drivers' safety. In the first place, women are more likely to suffer from fatigue compared with men. A study found that women in general have 1.4 times higher chance of complaining of fatigue than men {[}23{]}. However, females are found to have longer sleeping hours than their male counterparts of the same race {[}24{]}. In that study, it was found that the mean sleep hours for white females was 6.7 hours compared with 6.1 hours for white males, and 5.9 hours for black female compared with 5.1 hours for black males even after adjusting for socioeconomic status, lifestyle and sleep apnea {[}24{]}. Gender differences are huge in terms of working conditions. Females had significantly fewer working hours per week, with 47 hours versus 52 hours per week {[}25{]}. In general, women tend to work fewer hours within a week but are more prone to feel fatigue and have a higher risk of traffic incidences.

\hypertarget{traffic}{%
\subsection{Traffic}\label{traffic}}

\hypertarget{weather}{%
\subsection{Weather}\label{weather}}

Weather has both direct and indirect effects on drivers' safety critical events. On one hand, the increase of ambient temperature places risks on drivers' occupational safety, and possibly leads to cognition loss, heat stroke, and impairment of wakefulness. Evidence showed that the risk of mistakes and safety critical events increase in hot weather {[}26, 27{]}. Leard and Roth found that for a day with temperature above 80◦F there is a 9.5\% increase in fatality rates compared with a day at 50-60◦F {[}28{]}. A literature review found that 11 out of 13 studies indicated an increase in unintentional injuries associated with high temperatures {[}29{]}. On the other hand, real-time extreme weather conditions such as heavy rain, fog, storm, and snow can either impair the driver's visual capability or reduce the safety of driving on the road {[}30 - 32{]}. It is to noted that the cumulative time of driving in such extreme weather conditions could increase the chances of safety critical events. Studies that explore the association between precipitation and driving safety critical events consistently find a negative relationship. The positive linear relationship between precipitation and traffic accidents can be observed in both driver accidents and pedestrian accidents {[}31, 33{]}. Abdel-Aty et al.~used detector and sensor data to successfully predict more than 70\% of accidents with low visibility conditions {[}34{]}. The common problem for the literature exploring the relationship between ambient weather and safety driving critical events is the failure to include the cumulative effect of weather conditions. Instead, they all use an indicator variable to represent whether extreme weather happened during the trip or not, which could lead to potential bias in prediction models.

\hypertarget{hierarchical-models}{%
\section{Hierarchical models}\label{hierarchical-models}}

\hypertarget{bayesian-models}{%
\section{Bayesian models}\label{bayesian-models}}

\href{https://blog.csdn.net/u013841458/article/details/82495450}{Stochastic Gradient HMC}

\hypertarget{conceptual-framework}{%
\section{Conceptual framework}\label{conceptual-framework}}

\citet{cantor2010driver} suggested three factors that cause truck crashes: driver factors, vehicle factors (type and condition), and environmental factors.

\citet{roshandel2015impact} proposed five factors that affect traffic safety: (a) behavioral characteristics of the driver, e.g., impairment, fatigue, distractions; (b) vehicle --- the condition of the vehicle; (c) traffic --- the traffic conditions; (d) geometry --- geometric characteristics of the road, e.g.~curve, hill, ramps, etc.; and (e) environmental --- characteristics of the surrounding environment, such as weather conditions (rain, snow, night-time driving, etc.). Traffic conditions are the most studied of these and we focus on discussing them in this subsection.

\hypertarget{gaps-in-literature}{%
\section{Gaps in literature}\label{gaps-in-literature}}

\begin{itemize}
\tightlist
\item
  A focus on crashes instead of precursors of crashes
\item
  A focus on road segments rather than drivers
\item
  A focus on case-control comparison given the rareness of truck crashes rather than rates
\end{itemize}

\hypertarget{research-aims}{%
\section{Research aims}\label{research-aims}}

\textbf{Aim1}:

\textbf{Aim2}:

\textbf{Aim3}:

\hypertarget{methods}{%
\chapter{METHODS}\label{methods}}

\hypertarget{data-source}{%
\section{Data source}\label{data-source}}

\hypertarget{study-design}{%
\section{Study design}\label{study-design}}

\hypertarget{analytical-plan-for-aim-1}{%
\section{Analytical Plan for Aim 1}\label{analytical-plan-for-aim-1}}

\hypertarget{analytical-plan-for-aim-2}{%
\section{Analytical Plan for Aim 2}\label{analytical-plan-for-aim-2}}

\hypertarget{analytical-plan-for-aim-3}{%
\section{Analytical Plan for Aim 3}\label{analytical-plan-for-aim-3}}

\hypertarget{aim-1}{%
\chapter{AIM 1}\label{aim-1}}

\hypertarget{aim-2}{%
\chapter{AIM 2}\label{aim-2}}

\hypertarget{aim-3}{%
\chapter{AIM 3}\label{aim-3}}

\hypertarget{discussion}{%
\chapter{DISCUSSION}\label{discussion}}

\cleardoublepage

\bibliographystyle{plainnat}
\bibliography{bib/bib}

\cleardoublepage

\hypertarget{appendix-appendix}{%
\appendix}


\hypertarget{this-is-one}{%
\chapter{This is one}\label{this-is-one}}

\hypertarget{this-is-two}{%
\chapter{This is two}\label{this-is-two}}







\backmatter
\printindex


\end{document}
